\documentclass[12pt]{article}
\usepackage{graphicx}
\usepackage{makeidx}
\usepackage{amsmath}
\usepackage{amsfonts}
\usepackage{color}
\usepackage[all]{xy}


%%%%%%%%%%%%%%%%%%%%%%%%%%%%%%%%%%%%%%%%%%%%%%%%%%%%%%%%%%%%
\long\def\remove#1{}
\newcommand{\dom}{\mbox{dom}}

\newcommand{\vp}{\varphi}


\newcommand{\coNP}{\mbox{coNP}}

\newcommand{\NP}{\mbox{NP}}
\newcommand{\DP}{\mbox {D}^P}

\newcommand{\LeKl}{\mbox{$\sqcup$ }}
\newcommand{\card}[1]{\mbox{\#}(#1)}

\newcommand{\sss}{\mbox{\bf S}}
\newcommand{\jjj}{\mbox{\bf J}}

\newcommand{\QR}[1]{\mbox{$\ \mid\!\!\!\frac{#1}{\
    \stackrel{\mbox{\scriptsize\it Q-Res}}{\ }\ }\ $}}
\newcommand{\QUR}[1]{\mbox{$\ \mid\!\!\!\frac{#1}{\
    \stackrel{\mbox{\scriptsize\it Q-Pos-Unit-Res}}{\ }\ }\ $}}
%\baselineskip 0.2in

\newcommand{\AK}{\mbox{${\cal{A}}_K$}}

\newcommand{\AMB}{\mbox{$^A$\hspace{-0.5mm}$\models^B$}}
\newcommand{\AM}{\mbox{$^A$\hspace{-0.5mm}$\models$}}

\newcommand{\AEQB}{\mbox{$^A$\hspace{-0.5mm}$\approx^B$}}
\newcommand{\AEQ}{\mbox{$^A$\hspace{-0.5mm}$\approx$}}

\newcommand{\pbox}{\hbox to 6pt{\leaders\hrule width 6pt height 6pt\hfill}}

\newtheorem{definition}{Definition}[section]
\newtheorem{theorem}{Theorem}[section]
\newtheorem{lemma}{Lemma}[section]
\newtheorem{corollary}{Corollary}[section]
\newtheorem{proposition}{Proposition}[section]
\newtheorem{example}{Example}[section]
\newtheorem{remark}{Remark}[section]
\newenvironment{proof}{\parindent=0pt{\bf Proof: }}{
   \hspace*{\fill}\hbox to 6pt{\leaders\hrule width 6pt height 6pt\hfill}\par}


\pagestyle{plain}

\begin{document}
\section{Implication}
\begin{definition}
Let $\alpha$ be a formula over the variables $X \cup A$ and let $\beta$ be a formula over the variables $Y \cup B$.\\
We write $\alpha $ \AMB $\beta$ ($\alpha $\AM $\beta$ resp.) if and only if
$(\exists X \alpha) \models (\exists Y \beta)$ ($\exists X \alpha \models \beta$ resp.).
\end{definition}

\color{red}
\noindent Remark: When we use $X,Y$ and $A, B$, we have to specify them. Then a problem occurs that are they fixed for all $\alpha, \beta$ or they depends on $\alpha, \beta$ (I think they can change). So I suggest the following notations.\\

%\noindent {\bf Definition 1.1}\
%Given two formulas $\alpha,\beta$, let $X=\mbox{var}(\alpha)-\mbox{var}(\beta)$ and $Y=\mbox{var}(\beta)-\mbox{var}(\alpha)$. We define
%\begin{enumerate}
%\item $\alpha\, ^\exists\!\!\models^\exists \beta$ if and only if $\exists X\alpha\models \exists Y\beta$

%\item $\alpha\, ^\exists\!\!\models \beta$ if and only if $\exists X \alpha\models \beta$.
%\end{enumerate}
\color{black}


\begin{definition}
 Let $R$ and $T$ be classes of formulas. Then we define\\
{\em Problem:  $R \Longrightarrow_{conf} T$-Implication for $\models$, \AM, \AMB}\\
{\em Instance}: A set of components $C=\{\alpha_1,\cdots,\alpha_n\} \subseteq R$ over the variables $X \cup A$ and the target formula $\beta \in T$ over the variables $Y \cup B$. In order to avoid renaming we assume that $Y \cap A= X \cap B=$ is empty.\\
{\em Query}:
 $ \exists K \subseteq C: K \in$ SAT and $K \models \beta$ $((\exists X K)  \models \beta, \
(\exists X K) \models (\exists Y \beta)$ respectively)?\\
\end{definition}

\color{red}
\noindent {\bf Definition 1.2}\ Let $R$ and $T$ be two classes of formulas. Then we define the configuration implication problem
$(R, T, \models)$ (resp. $(R, T, ^\exists\!\models)$, $(R, T,^\exists\!\models^\exists$)) as follows:

\begin{description}
%\item {Problem:} $(R, T, \models)$ (resp. $(R, T, ^\exists\!\models)$, $(R, T,^\exists\!\models^\exists$))
\item {Instance:} A set of components $C=\{\alpha_1,\cdots, \alpha_n\}\subseteq R$ and a target formula $\beta\in T$, and two sets $X\subseteq \mbox{var}(C)-\mbox{var}(\beta)$ and $Y\subseteq \mbox{var}(\beta)-\mbox{var}(C)$.
\item {Query:} Does there exist $K\subseteq C$ such that $K\in \mbox{SAT}$ and $K\models \beta$ (resp. $\exists X K\models \exists Y \beta$, $\exists X K
\models \beta$)?
\end{description}
\color{black}


\begin{theorem}
The table contains the computational complexities of\\ $R \Longrightarrow_{conf} T$-Implication problems.\\
\begin{tabular}{|l|l|l|l|l|}
\hline
 & {\em Classes} &  $\models$ & \AM & \AMB \\ \hline
1. & DHORN $\Rightarrow_{conf}$ DHORN          & PTIME & PTIME  & coNP-c         \\ \hline
2. & DHORN $\leq k$ $\Rightarrow_{conf}$ DHORN & NP-c  & NP-c   & ---         \\ \hline
3. & 2-Term $\Rightarrow_{conf}$ Term          & NP-c  & NP-c   & NP-c           \\ \hline
4. & HORN $\models$ HORN                       & NP-c  & NP-c   & open $\Sigma^p_2$-c \\ \hline
5. & CLAUSE $\Rightarrow_{conf}$ LITERAL       & $\Sigma_2^P$-c & $\Sigma_2^P$-c & $\Sigma_2^P$-c \\ \hline
6. & CNF $\Rightarrow_{conf}$ CLAUSE, LITERAL  & $\Sigma_2^P$-c & $\Sigma_2^P$-c & $\Sigma_2^P$-c \\ \hline
7. & CNF $\Rightarrow_{conf}$ CNF              & $\Sigma_2^P$-c & $\Sigma_2^P$-c & open $\Sigma_3^P$-c \\ \hline
\end{tabular}
\end{theorem}

\color{red}
\noindent If you agree the notation. The above table would be.\\

\noindent\begin{tabular}{|l|l|l|l|l|l|}
\hline
 & {Class $R$ } & Class $T$ &  $(R,T,\models)$ & $(R, T, ^\exists\!\models)$ & $(R, T,^\exists\!\models^\exists$) \\ \hline
1. & DHORN & DHORN          & PTIME & PTIME  & coNP-c         \\ \hline
2. & DHORN $\leq k$ & DHORN & NP-c  & NP-c   & ---         \\ \hline
3. & 2-Term & Term          & NP-c  & NP-c   & NP-c           \\ \hline
4. & HORN & HORN                       & NP-c  & NP-c   & open $\Sigma^p_2$-c \\ \hline
5. & CLAUSE & LITERAL       & $\Sigma_2^P$-c & $\Sigma_2^P$-c & $\Sigma_2^P$-c \\ \hline
6. & CNF & CLAUSE, LITERAL  & $\Sigma_2^P$-c & $\Sigma_2^P$-c & $\Sigma_2^P$-c \\ \hline
7. & CNF & CNF              & $\Sigma_2^P$-c & $\Sigma_2^P$-c & open $\Sigma_3^P$-c \\ \hline
\end{tabular}

\color{black}




\vspace*{5mm}

\begin{proof}
Ad 1: (DHORN, $\models$ and \AEQ): Since every definite Horn formula is satisfiable, we only we have to test whether for the complete set of components $C$ we have $C \models \beta$ or $\exists X C \models \beta$. That can be performed in polynomial time.\\

(DHORN,\AMB): well-known trick\\

Ad 2: (DHORN $K$, $\models$ and \AM): Next we show the NP-completeness of the problem whether for a set of components $C$, a target formula $\beta$ in DHORN, and a fixed $k$ there exists some $K \subseteq C$ with $K \models \beta$ and at most $k$ components.\\
Since $K \models \beta$ is in polynomial time decidable, the problem is in NP (guess a subset $K \subseteq C$).
The NP-hardness can be shown by a reduction to the Minimal Input Set Problem (MI) for definite Horn formulas.
Let $\alpha$ be a definite Horn formula without unit clauses, and let $H$ be the set of variables occurring only
negatively in $\alpha$, and $y$ be a variable which occurs positively in $\alpha$. Then we have to decide whether
there exists a subset $I \subseteq H$ with at most $k$ variables, such that $I \wedge \alpha \models y$. This problem is known to be NP-complete \cite{mi}.

We can associate to $\alpha$ and $y$ a configuration problem as follows: We define $C = H \cup \{\alpha\}$, the target
formula is $y$. If there is a solution for the MI-problem then there is subset $K \subseteq C$ with at most $k+1$ elements
and $K \models y$. For the other direction, suppose there exists $K \subseteq C, |K | \leq k+1$ and $K \models y$.
\end{proof}

\begin{proof}
AD 3:  (Term,$\models$, \AM, \AMB): The components are terms with at most 2 literals and target formula is a term or an existentially quantified term, which can be reduced to a term. Obviously, the problems are in NP (guess $K$, the test
is simple). Now we will show the NP-hardness.

Let $\alpha = \alpha_1 \wedge \ldots \wedge \alpha_m$ be a 3-CNF formula over the variables $x_1, \ldots, x_n$. We associate
to $\alpha$ a set of 2-term components $C$, a target formula $\beta= q_1 \wedge \ldots \wedge q_m$, such that $\alpha$ is satisfiable if and only if
there exists $K \subseteq C$ for which $K$ is satisfiable and $K \approx\beta$.

Let $q_1, \ldots, q_m$ be new variables for $\alpha$.\\
We define $C=\{(q_i \wedge L) : 1 \leq i \leq m, L \in \alpha_i\}$. The variable $q_i$ indicates that $L$ occurs in the clause $\alpha_i$.\\
Suppose, $\alpha$ is satisfiable. Then there is a truth assignment $v$ with $v(\alpha)=1$ and therefore for every clause $\alpha_i$ a literal $L$ with $v(L)=1$.\\
We define $K = \{(q_i \wedge L) : 1 \leq i \leq m, v(L)=1, L \in \alpha_i\}$. Then $K$ is satisfiable and since
for every clause $\alpha_i$ there exists some $L \in \alpha_i$ with $v(L)=1$ we obtain $K \models q_1 \wedge \ldots \wedge q_m$.\\
For the other direction we assume that there is some $K \subseteq C$ with $K \in$ SAT and $K \models \beta$.
Then for every variable $q_i$ there is a term $q_i \wedge L$ in $K$. Since $K$ is satisfiable, the conjunction of literals $L$ of the components (2-terms) are satisfiable. This leads to a satisfying truth assignment for $\alpha$.
\end{proof}

\begin{proof}
(HORN, $\models$, \AM):\\
Since every term is a Horn formula, the NP-hardness follows from part (3).
 Whether or not $\exists X K) \models \beta$ can be decided in polynomial time. Therefore, we only have to guess a satisfiable subset $K$ of $C$. That shows that the problems belong to NP.\\

(HORN, \AMB):  That the problems are in $\Sigma^p_2$ is obvious. We can guess some satisfiable $K \subseteq C$ and
then to decide the coNP-problem $(\exists X K \models (\exists Y \beta)$. For the complementary problem guess a
truth assignment $v$ for the free variable and check whether $v(\exists X K)=1$ and $v(\exists Y \beta)=0$.
That can be performed in polynomial time , because we are dealing with quantified Horn formulas. Altogether,
we have $\Sigma^p_2$-procedure.\\

Now we have to how the $\Sigma^p_2$-hardness.

\color{red}

Consider an arbitrary $\Phi:=\exists U\forall W (\theta\vee\sigma)$ such that $\theta, \sigma\in$ 3-DNF, and that $\theta$ is negatively monotone
while $\sigma$ is positively monotone.
%
Let $$\begin{array}{l}\theta=\bigvee\{(\neg L_{i,1}\wedge \neg L_{i,2}\wedge \neg L_{i,3})\mid i=1,\cdots,n\},\\ \sigma=\bigvee\{(K_{j,1}\wedge K_{j,2}\wedge K_{j,3})\mid j=1,\cdots,m\},\end{array}$$
where $L_{i,p}$ and $K_{j,q}$ are variables from $U\cup W$.\\

We pick new variables $a_1,\cdots, a_n$, and $b_1,\cdots, b_m$, and $c$. Define

$$\begin{array}{cll}C_1&:=&\{a_1\wedge\cdots\wedge a_n\wedge b_j\rightarrow c\mid 1\leq j\leq m\}\cup\\
&&\{L_{i,p}\rightarrow a_i\mid 1\leq i\leq n, 1\leq p\leq 3\}\cup\\
&&\{K_{j,1}\wedge K_{j,2}\wedge K_{j,3}\rightarrow b_j\mid 1\leq j\leq m\}\\
C_2&:=&U\cup\{\neg u\mid u\in U\}\\
C&:=&C_1\cup C_2\\ \\
\beta&:=&\{a_1\wedge\cdots\wedge a_n\rightarrow c\}\cup\\
&&\{L_{i,p}\rightarrow a_i \mid 1\leq i\leq n, 1\leq p\leq 3\}
\end{array}$$

Let
$$X:=\{a_1,\cdots, a_n, b_1,\cdots, b_m\}, \ \ Y:=\{a_1,\cdots, a_m\}$$


Next we shall show: $\Phi$ is true iff $\exists K\subseteq C$ satisfiable $\exists X K\models \exists Y\beta$\\

Suppose $\Phi$ is true, then there is a truth assignment $t$ on $U$ such that $\forall W(\theta\vee \sigma)[t]$ is true.
Define
$$K:=C_1\cup\{u\mid t(u)=1, u\in U\}\cup\{\neg u\mid t(u)=0, u\in U\}$$
Obviously, $K$ is satisfiable. We show that $\exists X K\models\exists X\beta$. Consider any truth assignment $s$ satisfying $\exists X K$. Then clearly $s$ and $t$ agree on $U$.
There must be an extension $s'$ of $s$ such that $s'$ satisfies $K$.  Since  $\forall W(\theta\vee \sigma)[t]$ is true, we have two cases:\\

{\bf Case 1.} $\neg L_{i,1}\wedge\neg L_{i,2}\wedge \neg L_{i,3}$ is true under $s'$ for some $i$. Then we modify $s'$ to obtain $s''$ as follows:
%
$s''(a_i)=0, s''(a_h)=s'(a_h)$ for $h\not=i$, and $s''(c)=s'(c)$. It is easy to see that $s''$ satisfies $\beta$. Therefore $s$ satisfies $\exists Y\beta$ since $s, s', s''$ agree on variables in $U\cup W\cup\{c\}$.\\

{\bf Case 2.} Not Case 1. There must be some $j$ such that $K_{j,1}\wedge K_{j,2}\wedge K_{j,3}$ is true under $s'$. Then $s'(b_j)=1$. Because Case 1 does not happen, all $a_1,\cdots,a_n$ are true under $s'$. Hence, $s'(c)=1$. Consequently $s'$ satisfies $\beta$. Thus, $s$ satisfies $\exists Y\beta$.\\


Altogether, we have $\exists X K\models \exists Y \beta$. \\


For the inverse direction we suppose there is $K\subseteq C$ such that $K$ is satisfiable and $\exists X K\models \exists Y \beta$. \\

Please note that for any satisfiable $K'\subseteq C_2$, $C_1\cup K'$ is satisfiable. Thus we may assume that $C_1\subseteq K$.\\


{\bf Case 1.} Either $u\in K$ or $\neg u\in K$ for any variable $u\in U$. Let $t$ be the truth assignment defined by $t(u)=1$ iff $u\in K$. We shall show $\forall W(\theta\vee \sigma)[t]$ is true. Suppose otherwise, then $t$ has an extension $t'$ which makes $\neg\theta \wedge \neg \sigma$ true.
That is, for all $i=1,\cdots, n$ there is some $p=1,2,3$ such that $L_{i,p}$ is true under $t'$, and for all $j=1,\cdots,m$ there is some $q=1,2,3$ such that $\neg K_{j,q}$ is true under $t'$. Now we may extend $t'$ to $t''$ as follows: $t''(a_1)=\cdots=t''(a_n)=1$, $t''(b_1)=\cdots=t''(b_m)=0$, $t''(c)=0$. Clearly, $t''$ makes $K$ true. Let $s$ be the assignment obtained from $t$ by setting $c$ to be false. Then $s$ satisfies $\exists X K$ since $s$ and $t''$ agrees on $U\cup W\cup\{c\}$. However, $s$ can not be extended to satisfy $\beta$ because any extension of $s$ must make $a_1,\cdots, a_n$ and $c$ be true if it satisfies $\beta$. That is, $s$ does not satisfy $\exists Y\beta$ contradicts the assumption $\exists X K\models\exists Y\beta$.
Therefore, $\forall W(\theta\vee \sigma)[t]$ is true, so is $\exists U\forall W(\theta\vee \sigma)$.\\

{\bf Case 2.} Not Case 1. Let $U_1=:\{u\mid u\in K \mbox{ or }\neg u\in K\}$. Let $W_1:=W\cup (U-U_1)$. Now we consider $\exists U_1\forall W_1(\theta\vee\sigma)$. Then by the proof of Case 1, we obtain that $\exists U_1\forall W_1(\theta\vee\sigma)$ is true. Since $U_1\subseteq U, W\subseteq W_1$ it follows that $\exists U\forall W(\theta\vee\sigma)$ is also true.

\color{black}

\end{proof}



\vspace*{3mm}




\begin{proof}
AD 5 and 6: (CNF, CLAUSE $\Longrightarrow_{conf}$ CLAUSE, Literal; \AMB)\\
At first we show that the problems are in $\Sigma^p_2$. It suffices to prove this upper bound for \AMB
and the case that the components are CNF-formulas and the target formula is a clause
$\beta =(L_1 \vee \ldots \vee L_m)$. Since $\exists Y \beta$ can be reduced to true, we can distinguish two cases:

Case 1: A literal $L$ over some variable $y \in Y$ occurs in $\beta$. Then we have $\exists Y \beta = 1$. That means our configuration problem is the question whether
$\exists K \subseteq C: (\exists X K) \models 1$, which is always true.\\

Case 1: $\beta$ contains no $Y$-variables, then we obtain $\exists Y \beta = \beta$ and we proceed with $\beta$.\\
$\exists K \subseteq C: K$ is satisfiable and $(\exists X K) \models \beta$ iff
$\exists K \subseteq C: K \in \overline{\mbox{SAT}}$ and $\forall X (\neg K \vee \beta)$ is a tautology.
That leads to $\Sigma^p_2$-procedure: Guess a satisfiable $K$ and check the tautology.\\

Now it remains to prove the $\Sigma^p_2$-hardness for components given  as clauses and a target formula given as a literal.
At first we show the poly-time equivalence of the configuration problem where the components are CNF-formulas and
the problem where the components are clauses.
Let $\beta=z$ be the target formula.\\
Let $C=\{\alpha_1, \ldots, \alpha_n\}$ be a set of components, where $\alpha_i$ are CNF-formulas.
For each formula $\alpha_i= \alpha_{i,1} \wedge \ldots \wedge \alpha_{i,t_i}$  we introduce new variables $q^i_0, \ldots, q^i_{t_i}$ and associate to $\alpha_i$ the components $q^i_0, (\neg q^i_0 \vee \alpha_{i,1} \vee q^i_1), \ldots
(\neg q^i_{t_i-1} \vee \alpha_{i,t_i} \vee q^i_{t_i}), \neg q^i_{t_i}$.

Let $C_{cl}$ be the set of these clauses. Then there exists $K \subseteq C$: $K \in$ SAT and $K \models z$ if and only if
there exists $K^* \subseteq C^*$: $K^* \in$ SAT and $K^* \models z$.\\

Now it remains to show the $\Sigma_2^P$-hardness:\\
Let $\Phi:= \exists x_1, \ldots, x_n \forall y_1, \ldots, y_m \varphi$ be a closed formula, where
$\varphi$ is a propositional DNF-formula. For those formulas the satisfiability problem is $\Sigma^p_2$-complete.
For a new variable $z$ we define the set of components\\
$C= \{ \varphi \rightarrow z, x_1, \neg x_1, \ldots, x_n, \neg x_n\}$ and the target formula $\beta=z$. Please note, that $\varphi \rightarrow z$ can easily transformed into an equivalent CNF-formula of length linear in the length of
$\varphi$.

Then it holds:\\
$\Phi$ is true if and only if $\exists K \subseteq C: K \in$ SAT and $K \models z$.\\
From left to right:\\
Suppose $\Phi$ is true. Then there is a partial truth assignment $v(x_1)= \epsilon_1, \ldots v(x_n)= \epsilon_n$, such
that $\forall y_1 \ldots \forall y_m \varphi([x_1/\epsilon_1, \ldots, x_n/\epsilon_n])$ is true. The subset of components
$K= \{ \varphi \rightarrow z, x_1^{\epsilon_1}, \ldots, x_n^{\epsilon_n}\}$ is satisfiable. Assume that $K\not \models z$.
Then there is a truth assignment $v$ with $v(z)=0$, but $v(K)=1$. Then we obtain $v(\varphi([x_1/\epsilon_1, \ldots, x_n/ \epsilon_n]) \rightarrow z)=1$ and therefore $v(\varphi([x_1/\epsilon_1, \ldots, x_n/ \epsilon_n])=0$ as a contradiction.\\

From right to left: Suppose, $\exists K \subseteq C: K \in$ SAT and $K \models z$. Then $\varphi \rightarrow z$ is in $K$,
because $z$ occurs only in this component.
\end{proof}

\vspace*{3mm}
\begin{proof}
Ad 7: (CNF $\longrightarrow_{conf}$ CNF, $\models$, \AEQ):)
Because of part (6) it suffices to show that the problems belong to $\Sigma_2^p$.
(see part 5,6)\\
the hardness follows from (5,6)\\

(CNF $\Longrightarrow_{conf}$, \AEQB):\\
Membership $\Sigma_3^p$:\\

Next we show the $\Sigma^p_3$-hardness.\\




\end{proof}

%\section{Compare}
%\begin{theorem}
%Sible Formula:\\
%Implication\\
%\begin{tabular}{|l|l|l|l|l|}
%\hline
% & {\em Classes} &  $\models$ & \AM & \AMB \\ \hline
%1. & DHORN $\Rightarrow_{conf}$ DHORN          & PTIME & PTIME  & coNP-c         \\ \hline
%2. & 2-Term $\Rightarrow_{conf}$ Term          & PTIME & PTIME   & PTIME          \\ \hline
%3. & HORN $\models$ HORN                       & PTIME  & PTIME   & coNP-c \\ \hline
%4. & CLAUSE $\Rightarrow_{conf}$ LITERAL       & PTIME & PTIME & PTIME\\ \hline
%5. & CNF $\Rightarrow_{conf}$ CNF              & coNP-c & coNP-c & coNP-c \\ \hline
%\end{tabular}
%\end{theorem}
%\vspace*{5mm}



\section{Equivalence}


\begin{definition}
Let $\alpha$ be a formula over the variables $X \cup A$ and let $\beta$ be a formula over the variables $Y \cup B$.\\
We write $\alpha $ \AEQB $\beta$ ($\alpha $\AEQ $\beta$ resp.) if and only if
$(\exists X \alpha) \models (\exists Y \beta)$ ($\exists X \alpha \models \beta$ resp.).
\end{definition}





\begin{definition}
Let $R$ and $T$ be classes of propositional formulas. Then we define\\
{\bf Problem $R \Longrightarrow_{conf} T$-Equivalence for $\approx$, \AEQ, \AEQB}\\
{\bf Instance}: A set of components $C=\{\alpha_1,\cdots,\alpha_n\} \subseteq R$ over the variables $X \cup A$ and the target formula $\beta \in R$ over the variables $Y \cup B$.\\
{\bf Query}:
 $ \exists K \subseteq C: K \in$ SAT and $K \approx \beta$ (resp. $\exists X K \approx \beta, \
\exists X K \approx \exists Y \beta)$?\\
\end{definition}





\begin{theorem}
Equivalence\\

\begin{tabular}{|l|l|l|l|l|}
\hline
 & {\em Classes} &  $\approx$ & \AEQ & \AEQB \\ \hline
1. & DHORN $\Rightarrow_{conf}$ DHORN & PTIME & open & coNP-h\\ \hline
2. & 2-Term $\Rightarrow_{conf}$ Term & PTIME & NP-c & NP-c\\ \hline
3. & HORN  $\Rightarrow_{conf}$ HORN & PTIME & $\Sigma^p_2$-c &  $\Sigma^p_2$-c \\ \hline
4. & CL $\Rightarrow_{conf}$ LITERAL & coNP-c & $\Sigma^p_2$-c & $\Sigma^p_2$-c\\ \hline
5. & CNF $\Rightarrow_{conf}$ LITERAL & coNP-c  & $\Sigma^p_3$ & $\Sigma^p_3$-c \\ \hline
6. & CNF $\Rightarrow_{conf}$ CNF & $P^{NP[\text{log}\, n]}$, $D^P$-h  & $\Sigma^p_3$-c& $\Sigma^p_3$-c \\ \hline

\end{tabular}
\end{theorem}
\vspace*{5mm}


\begin{proof}
Ad 1: ($\approx$):\\

(\AEQ):





\end{proof}
\begin{proof}
AD 3: ($\approx$):\\
(\AEQ) and (\AEQB):\\
$$\Phi:=\exists y_1,\cdots, y_k \forall x_1,\cdots, x_m \left(c_1\vee\cdots\vee c_n \right)$$
%
Where $c_i$ is a conjunction of literals.\\

for each variable $z$, introduce a new variable $\pi(\neg z)$. Let $\pi(z)=z$.\\

introduce $U$.\\

Let $$\psi_0:=\left(\bigwedge_{j=1}^m (\neg \pi(\neg x_j)\vee \neg x_j)\right), \ \psi_1:= \left(\bigwedge_{i=1}^k(\neg \pi(\neg y_i)\vee \neg y_i)\right)$$

$$\theta:=\left(\bigvee_{i=1}^m (\neg x_i\wedge \neg\pi(\neg x_i))\right)$$

%$$\theta_i:=\left(\bigvee_{x\not\in c_i, \neg x\not\in c_i}(\neg x\wedge\neg\pi(\neg %x))\right)$$

Please note that $\theta$ is not HORN  when $n>1$. Forturnately, by using Tseiting algorithm, one can transform  $\theta$ to an equivalent HORN formula when restricted to old variables. \\



For a conjunction $c=L_1\wedge\cdots\wedge L_s$, we write
$\pi(c):=\pi(L_1)\wedge\cdots\wedge \pi(L_s)$


$$\begin{array}{lcl}C_0&:=&\{\psi_0\wedge\psi_1 \wedge\left( \bigwedge_{i=1}^n(\pi(c_i)\rightarrow U\vee \theta)\right)\}\\

C&:=&C_0\cup \{y_i,\pi(\neg y_i)\mid i=1,\cdots,k\}\\
\beta&:= &(U\vee\theta)\wedge \psi_0 \\
V&:=&\{U\}\cup\{x_1,\cdots,x_m,\pi(\neg x_1),\cdots, \pi(\neg x_m)\}
\end{array}$$

We shall show

$$(\exists K\subseteq C \text{ such that } K\equiv_V \beta) \text{ if and only if $\Phi$ is true}.$$

\ \\

\color{red}
!!!! I suddenly find that $U\vee \theta $ may not be HORN. Forturnately we can change $U$ to $\neg U$ in $C$ and $\beta$. Then the following proof should be changed accordingly
\color{black}\\

Suppose $\Phi$ is true. There is truth assignment
$e$ on $\{y_1,\cdots, y_m\}$ such that $\forall x_1,\cdots,x_m\varphi[\vec{y}/e]$ is true, where $\vec{y}=y_1\cdots,y_k$. Define
%
$$K=C_0\cup \{y_i\mid t(y_i)=1, 1\leq i\leq k\}\cup \{\pi(\neg y_i)\mid t(y_i)=0, 1\leq i\leq k\}$$

For any satisfying truth assignment $t$ for $K$, if it makes $\theta$ true then $\beta$ is true under $t$. Suppose $t$ makes $\theta$ false, then $t$ corresponds a truth assignment on $x_1,\cdots, x_n$ accoding the truth of $x_i$ and $\pi(\neg x_i)$. Since $\Phi$ is true, some  $\pi(c_i)$ must be true under $t$, then $T(U)=1$. Therefore, $t(\beta)=1$.

Suppose $s$ is a satisfying truth assignment for $\beta$. Then $s*e$ satisfies $\psi_0\wedge \psi_1$. If $s$ makes $\theta$ true, then $s*e$ satisfies $K$. Suppose $s(\theta)=0$. Then $s(U)=1$, hence $K$ is still satisfied by $s*e$.

Altogether, we have $K\equiv_V\beta$.\\


For the inverse direction, suppose there is satisfiable $K\subseteq C$ such that $K\equiv_V\beta$.


Clearly, the formula in $C_0$ must be in $K$. Then by formula $\psi_1$, either $y_i$ or  $\pi(\neg y_i)$ is not in $K$.

Pick a truth assignment $e$ on $\{y_1,\cdots, y_k\}$ such that if $y_i\in K$ then $e(y_i)=1$, and $e(y_i)=0$  if else.

We need to show $\forall x_1,\cdots,x_m(c_1\vee\cdots\vee c_n)$ is true.

Consider any truth assigment $s$ on $\{x_1,\cdots, x_m\}$. Let $s'$ be the assignment defined by $s'(\pi(\neg x_i)=1$ iff $s(x_i)=0$. Then $e'*s'$ satisfies $\psi_0\wedge\psi_1$, where $e'$ is obtained from $e$ is the same way as $s'$.
We claim that, $(e'*s')$ makes $\pi(c_i)$ true for some $i$. Otherwise, we could set $U$ to be false,  and get a truth assignment satisfying $K$ but bot satisfying $\beta$ (please note that $s'(\theta)=0$), contradict the $V$-quivalence.

Consequently, $\Phi$ is true.
 \end{proof}




\begin{proof}
Ad 4: ($\approx$):
Let $C=\{(\alpha_1 \vee L, \ldots, (\alpha_n \vee L), (\sigma_1 \vee \neg L), \ldots, (\sigma_r \vee \neg L), \pi_1, \ldots, \pi_t\}$ be a set of components, where $L$ does not occur in $\pi_j$. Futhermore, let $\beta=L$ be the target formula. Then we have:\\
$\exists K \subseteq C: K \in$ SAT and $K \approx L$ iff $\{\alpha_1 \vee L), (\alpha_n \vee L)\} \approx L$
iff $\alpha_n \wedge \ldots \wedge \alpha_n \in \overline{\mbox{SAT}}$. This shows that the problem is in coNP and because of the coNP-completeness of $\overline{\mbox{SAT}}$-problem the hardness, too.


(\AEQ) and (\AEQB):

$\Sigma^p_2$-hard\\
$\Phi= \exists X \forall Y \phi$ where $\phi \in$ DNF\\
$\phi = \bigvee_{1 \leq i \leq r}$\\

$C =\{\phi \rightarrow z, x_1, \neg x_1, \ldots, x_n, \neg x_n\}$, $\beta=z$\\

$\exists K: ( \exists X K \in$ SAT and $(\exists X K \approx z$\\
iff $\Phi$ is true\\
proof: \\

left to right: $\exists X K = \exists X:\{(\neg \phi_{i_1} \vee z), \ldots, (\neg \phi_{i_r} \vee z), x_1^{\epsilon_1}, \ldots, x_t^{\epsilon_t}\} \approx z$\\
$\Longrightarrow$\\
$\bigwedge_{1 \leq j \leq r} (\neg \phi_{i_j} \vee z)[x_1/\epsilon_1, \ldots, x_t/\epsilon_t] \in
\overline{\mbox{SAT}}$\\
$\Longrightarrow$\\
$\bigvee_{1 \leq j \leq r} (\neg \phi_{i_j} \vee z)[x_1/\epsilon_1, \ldots, x_t/\epsilon_t] \in
\overline{\mbox{TAUT}}$\\
$\Longrightarrow$\\
$\forall x_{t+1} \ldots \forall x_n \forall Y \bigcup_{1 \leq j \leq r} $ is true\\
$\Longrightarrow$\\
$\exists x_1 \ldots \exists x_n \forall Y \bigcup_{1 \leq j \leq r} $ is true\\
$\Longrightarrow$\\
$\exists X \forall Y \bigcup_{1 \leq j \leq r} \bigvee_{1 \leq j \leq r} \phi_j$ is true\\
$\Longrightarrow$\\
$\exists X \forall Y \phi$ is true.\\

right to left:
Suppose $\Phi$ is true for $x_1= \epsilon_1, \ldots, x_n= \epsilon_n$. Then we choose
$K=\{ (\neg \phi_1 \vee z), \ldots, (\neg \phi_r \vee z), x_1^{\epsilon_1}, \ldots, x_n^{\epsilon_n}\}$\\
$K$ is satisfiable\\
Then we have:
$\exists X K \approx \bigwedge_{1 \leq j \leq r} (\neg \phi_j \vee z)[x_1/\epsilon_1, \ldots, x_n/\epsilon_n]
\approx (\bigwedge_{1 \leq j \leq r} \neg \phi_j)[x_1/\epsilon_1, \ldots, x_n/\epsilon_n] \vee z
\approx  \neg (\bigvee_{1 \leq j \leq r} \phi_j[x_1/\epsilon_1, \ldots, x_n/\epsilon_n]) \vee z$\\
$\approx \neg (\exists X \forall Y \phi) \vee z)$ $\approx z$\\

Ende\\


$\exists x_1 \ldots \exists x_t \forall x_{t+1} \ldots \forall x_n \forall Y \bigcup_{1 \leq j \leq r} $







in $\Sigma^p_2$:

\begin{center}
$\exists K ((\exists X K) \wedge \neg L \in \overline{\mbox{SAT}}$ and
$L \wedge (\forall X \neg K) \in \overline{\mbox{SAT}}$\\
iff\\
$\exists K \{(\forall A [(\exists X K) \wedge \neg L]$ false and
$\forall A [L \wedge (\forall X \neg K)]$ false\\
iff\\
$\exists K \{(\exists A [(\forall X \neg K) \vee  L]$ true and
$\exists A [\neg L \vee (\exists X K)]$ true\\
iff\\
$\exists K \{(\exists A [(\forall X \neg K) \vee  L]$ true and
$\exists A [\neg L \vee (\exists X K)]$ true\\
iff\\
$\exists K \exists A' \exists X' \exists A \forall X [(\neg K \vee L) \wedge (\neg L \vee K')]$
\end{center}

differnce between CL and CNF\\
to show for CL:
\end{proof}


\begin{proof}
Ad 5: $\approx$:\\
Let $C=\{\alpha_1, \ldots, \alpha_n\}$ be the set of components with $\alpha_i \in$ CNF and the target formula $\beta=L$
Because of case (4), it remains to show that the problem is in coNP. Then we have:\\
$\exists K \subseteq C: K \in$ SAT and $K \approx L$ iff $\{\alpha_1 \vee L), (\alpha_n \vee L)\} \approx L$
iff $\alpha_n \wedge \ldots \wedge \alpha_n \in \overline{\mbox{SAT}}$.
\end{proof}

\begin{proof}
AD 5: {\em CL, CNF $\Rightarrow_{conf}$ CL}\\
Because of case 4, it suffices to show that for CNF-components the problems belong to the desired complexity classes.
Let $C=\{\alpha_1,\cdots,\alpha_n\} \subseteq R$ over the variables $X \cup A$
and the target formula $\beta \in R$ over the variables $Y \cup B$, where $\alpha_i$ CNF-formulas and $\beta$
is a clause.\\
($\approx$): Let $C=\{\alpha_1, \ldots, \alpha_n\}$ be the set of components. We define
$L(\beta)=\{\alpha_i : \beta \models \alpha_i\}$. Please note, that $\beta \models \alpha_i$ can be decided in polynomial time, since $\beta$ is a clause and $\alpha_i \in$ CNF. Then we have: $L(\beta) \models \beta$ iff
$\exists K \subseteq L(\beta): K \approx \beta$ iff $\exists K \subseteq \beta$. If $L(\beta)$ \AEQ $\beta$ then
$L(\beta)$ is satisfiable. The decision problem $L(\beta) \models \beta$ is in coNP.\\
(\AEQ): We have to decide whether or not $\exists K \subseteq C: (\exists X K \approx \beta$.
$\exists X \alpha \approx \beta$ is in $\Sigma^p_2$

\end{proof}
\begin{proof}
Ad 1, 2, 3: Since every DHORN formula and every term is a Horn formula, it suffices to prove the polynomial time solvability for the Horn case.\\
Let $C=\{ \alpha_1, \ldots, \alpha_n\}$ be the set of components and $\beta$ the target formula. We define
$L(\beta) =\{\alpha_i : \beta \models \alpha_i, 1 \leq i \leq n\}$. If $\beta$ is unsatisfiable  then there is no solution. Otherwise, $L(\beta)$ is satisfiable.
There exists $K \subseteq C: K \in$ SAT and $K \approx \beta$ if and only if $L(\beta) \models \beta$. In other words
$L(\beta)$ is a solution, if there exists one. That can be seen as follows: Suppose $K $ is a solution. Then
$K \subseteq L(\beta)$ and $K \models \beta \models L(\beta)$. Since $K \subseteq L(\beta)$ we obtain
$L(\beta) \approx \beta$.
\end{proof}


\begin{proof}
Ad 4 and 5: For components in form of CNF-formulas and target formulas given as a clause. The configuration problem is in coNP.

At first we show the coNP-hardness. Let $\alpha= \alpha_1 \wedge \ldots \wedge \alpha_n$ be a CNF-formula and $x$ a new variable. Then $\sigma= (\alpha_1 \vee x) \wedge \ldots \wedge (\alpha_n \vee x)$ is equivalent to $x$ if and only if
$\alpha$ is unsatisfiable. Now we associate to $\sigma$ a set of components $C=\{(\alpha_1 \vee x), \ldots,(\alpha_n \vee x)\}$ and the target formula $x$. Then it holds:\\
$\exists K \subseteq C: C \in$ SAT and $K \approx x$ iff $\alpha \not \in$ SAT. If $\alpha \not \in$ SAT then $K:= C$
is a solution. For the other direction let $K=\{\alpha_{i_1}, \ldots \alpha_{i_m}\} \subseteq C$ be a solution.
Since  $K \approx x$ the formula $\alpha_{i_1} \wedge \ldots \wedge \alpha_{i_m}$ is unsatisfiable and therefor $\alpha$ is unsatisfiable.
\end{proof}

Let $\alpha= \alpha_1 \wedge \ldots \wedge \alpha_n$ be a formula in CNF.  For $1 \leq i \leq n$ we introduce a renaming, which substitutes every variable $y$ in $\alpha$ with $y^i$. The new clauses are denoted as $\alpha^i_j$. For a new variable $x$
we associate to $\alpha$ the set of components
$C_0 =\{(\alpha_1 \vee x), \ldots, (\alpha_n \vee x)\}$ and for $1 \leq i \leq n C_i=\{\alpha^i_j : 1 \leq j \leq n, j\not = i\}$. Then for the target formula $\beta = x \wedge C_1 \wedge \ldots \wedge C_n$ it holds\\
$\alpha$ is minimal unsatisfiable if and only if $\exists K \subset C_0 \cup \bigcup_{1 \leq i \leq n} C_i:
K \in$ SAT and $K \approx \beta$.

If $\alpha$ is minimal unsatisfiable then $\alpha $ is unsatisfiable and therefore $C_0 \approx x$. Since
after the deletion of a clause in $\alpha$ the resulting formula is satisfiable, $K = C_0 \cup \bigcup_{1 \leq i \leq n} C_i$ is satisfiable and equivalent to $\beta$. For the other direction let $K$ be a solution. Since $K \models x$
and $K$ is satisfiable, we see that $\alpha$ is  unsatisfiable. More over, because $C_i$ is part of the target
formula after the deletion of a clause in $\alpha$ the formula is satisfiable for every $1 \leq i \leq n$. Hence,
$\alpha$ is minimal unsatisfiable.



\section{Restricted Equivalence}

\begin{definition}
Let $\alpha$ be a formula over the variables $X \cup A$ and let $\beta$ be a formula over the variables $Y \cup B$.\\
1. We say that $\alpha$ is {\em $(A,B)$-equivalent} to $\beta$,
in symbols $\alpha \approx^{A,B} \beta$, if and only if
$(\exists x \in X: \alpha) \approx (\exists y \in Y: \beta)$.

\noindent 2. We say
$\alpha$ {\em $A$-equivalent} to $\beta$,
in symbols $\alpha \approx^{A} \beta$, iff
$(\exists x \in X: \alpha) \approx \beta$.\\
\end{definition}


\begin{theorem}
Restricted-Equivalence\\

\begin{tabular}{|l|l|l|}
\hline
 & {\em Classes} &  $\approx^{A,B}$\\ \hline
1. & 2-Term, Term $\Rightarrow_{conf}$ Term & NP-complete \\ \hline
2. & DHORN $\Rightarrow_{conf}$ DHORN & PTIME \\ \hline
3. & HORN $\Rightarrow_{conf}$ HORN & $\Sigma_2^P$-complete \\ \hline
4. & CLAUSE $\Rightarrow_{conf}$ LITERAL &  $\Sigma_3^P$-complete\\ \hline
5. & CNF $\Rightarrow_{conf}$ CNF & $\Sigma_3^P$-complete\\ \hline
\end{tabular}
\end{theorem}
\vspace*{5mm}
\begin{proof}
AD 1:  Let $\alpha = \alpha_1 \wedge \ldots \wedge \alpha_m$ be a 3-CNF formula over the variables $x_1, \ldots, x_n$. We associate
to $\alpha$ a set of components $C$, a target formula $\beta$, and a set of variables $R$, such that $\alpha$ is satisfiable if and only if
There exists $K \subseteq C$ for which $K$ is satisfiable and $K \approx_R\beta$.

Let $q_1, \ldots, q_m$ be new variables.\\
We define $C=\{(q_i \wedge L) : 1 \leq i \leq m, L \in \alpha_i\}$, $\beta = q_1 \wedge \ldots \wedge q_m$, and
$R=\{q_1, \ldots, q_n\}$.\\
Suppose, $\alpha$ is satisfiable. Then there is a truth assignment $v$ with $v(\alpha)=1$.\\
We define $K = \{(q_i \wedge L) : 1 \leq i \leq m, v(L)=1, L \in \alpha_i\}$. Then $K$ is satisfiable and since
for every clause $\alpha_i$ there exists some $L \in \alpha_i$ with $v(L)=1$ we obtain $K \approx_R q_1 \wedge \ldots \wedge q_m$.\\
For the other direction we assume that there is some $K \subseteq C$ with $K \in$ SAT and $K \approx_R \beta$.
Then for every variable $q_i$ there is a term $q_i \wedge L$ in $K$. Since $K$ is satisfiable, the conjunction of literals $L$ of the components (2-terms) are satisfiable. This leads to a satisfying truth assignment for $\alpha$.
\end{proof}

\begin{proof}
AD 3: see $\models$
\end{proof}

\begin{proof}
the problem of determining whether $F\equiv_V G$  is in $\Pi_2^P$ (for the complementary problem: we guess a clause $\gamma$ over $V$, check that $G\models \gamma$ but $F\not\models \gamma$, or $G\not\models\gamma$ but $F\models\gamma$).

Thus, R-equivalence configuration is $\Sigma_3^P$ (guess a $K\subseteq C$, check that it is satisfiable and $K\equiv_V \beta$.) \\


Next we show the hardness.

Consider $\Phi:=\exists \vec{x}\forall\vec{y}\exists \vec{z} \varphi$ where $\varphi$ is a CNF formula. \\

We assume w.l.o.g. that $\varphi$ contains a non-tautological clause over $\vec{z}$ (otherwise, we pick new varaiable $z'$ and consider $\exists\vec{x}\forall\vec{y}\exists\vec{z}\exists z'(\varphi\wedge z'$)).

By this assumption, $\forall\vec{x}\exists\vec{y}\exists\vec{z}\neg \varphi$ is true\\

We assume that each clause contains a positive occurrence of a variable from $\vec{y}\cup\vec{z}$. If otherwise pick a new variable $u$, and consider $\exists \vec{x}\forall\vec{y}\forall u\exists \vec{z}\varphi^u$, where $\varphi^u$ is obtained from $\varphi$ by adding $u$ to clauses containing no positive literal from $\vec{y}\cup\vec{z}$. Clearly, $\Phi$ and the resulting formula has the same truth.

By this assumption, $\forall\vec{x}\exists\vec{y}\exists\vec{z}\phi$ is true. (we will use this later).\\

$\varphi$ can be written as $(c'_1\vee c''_1)\wedge\cdots\wedge (c'_n\vee c''_n)$ in which $c'_i$ is over $\vec{x}$, while $c''_i$ is over $\vec{y}\cup\vec{z}$.

Pick new variables $w_1, \cdots, w_n,w$. Let

Let $$\psi:= \left(\bigwedge_{i=1}^n (c'_i\rightarrow w_i)\right)\wedge
\left(\bigwedge_{i=1}^n(\neg c'_i\wedge \neg c''_i\rightarrow \neg w_i)\right)$$

Let $$C_0:=\{\psi\wedge \neg c'_1\wedge  (c''_1\rightarrow w_1) , \cdots, \psi\wedge\neg c'_n\wedge (c''_n\rightarrow w_n)\}$$

$$C_1=\{c'_1, \cdots, c'_n, \}$$

$$C_2=\{x_1,\neg x_1\cdots, x_n,\neg x_n\}$$

$$C_3=\{(w\rightarrow (c'_1\vee c''_1)\wedge\cdots\wedge (c'_n\vee c''_n))\}$$

Now let $$C:=C_0\cup C_1\cup C_2\cup C_3\cup\{w\}$$

Let $\beta$ be $$ w_1\wedge\cdots\wedge w_n\wedge w$$

Let $V:=\vec{y}\cup\{w_1,\cdots,w_n,\}\cup\{w\}$\\


%Claim: Suppose $K\subseteq C$ satisfiable, $c_i\not \in K$ and $K\not\models c_i$, %then
%$K\cup\{\neg c_i, \psi\wedge (c''_i\rightarrow w_i)\}$ is still satisfiable.


(Lemma: $F\models_V G$ iff for for any truth assignment $t$ on $V$, if it can be extended to a satisfying truth assignment for $F$, it can be extended to an satisfying truth assignment for $G$.
) \\

We shall show $\Phi$ is true if and only if there is satisfiable $K\subseteq C$ s.t. $K\equiv_V \beta$\\



\color{red} From right to left:\color{black}

Given a such $K\equiv_V \beta$.
Clearly, $K\cap C_0$ is non-empty.\\



It must be that $w\in K$.
Let $t$ satisfy $K$.
By our assumption we know $t\uparrow\vec{x}$ can be extended to $s$ which falsifies $\varphi$. Now we assign truth values to $w_i$ according to the tuth values $s(c'_i)$ and $s(c''_i)$ to make formulas $\psi$ and  $(c''_i\rightarrow w_i)$ to be true. There  must be some $w_i$ which is false. Then if we set $w$ to be false, $K$ is still be satisfied by $s$ (because clauses in $C_3$ are satisfied). This contradict the $V$-euivalence.\\

It must be that $C_3\subseteq K$, i.e., $w\rightarrow (c'_1\vee c_1'')\wedge\cdots\wedge (c'_n\vee c''_n)$ is in $K$. Suppose it is not the case. Similar as above, there would be a satisfying truth assigment of $K$ which makes some $w_i$ false, contradicts the $V$-equivalence. \\


Suppose $K\not\models c'_i$. Then $\psi\wedge\neg c'_i \wedge (c''_i\rightarrow w_i)$ must be in $K$.
Otherwise, $K\cup\{\neg c'_i\}$ is consitent,  then there is a satisfying truth assignment $t$ for $K$ such that $t(c'_i)=0, t(w)=1, t(w_i)=0$. (In fact by our assumption, $t$ can be assumed to satisfiy $(c'_1\vee c''_1)\wedge\cdots\wedge (c'_n\vee c''_n)$ because every clause contains positive literal from $\vec{y}\cup\vec{z}$). This contradict the fact that $K\equiv_V\beta$.


Now we can see for each $i$, either $K\models \neg c'_i$ or $K\models c'_i$.
That is, for any two satisfying truth assignments $t_1, t_2$ of $K$, each $c'_i$ has the same truth under $t_1\uparrow \vec{x}$ and $t_2\uparrow\vec{x}$.  \\





Now fix a trut assigment $e$ on $\vec{x}$ which can be extended to a satisfying truth assigment of $K$.

Consider any truth assigment $s$ on $\vec{y}$. Since $s$ can be extended to satisfy $\beta$, it can be extended to a truth assigment $t$ which satisfies $K$. Please note we can assmue that $t\uparrow\vec{x}$ is $e$.
Since $w$ and $w\rightarrow (c'_1\vee c''_1)\wedge\cdots\wedge (c'_n\vee c''_n)$ are in $K$. We can see $t$ satisfy $\varphi$.

Consequently, for any truth assigment $s$ on $\vec{y}$, $e*s$ can be extended to a satisfying truth assigment of $\varphi$. Thus, $\Phi$ is true.


\ \\

{\color{red} From left to right}.\color{black}


Suppose $\Phi=\exists\vec{x}\forall\vec{y}\exists\vec{z}\varphi$ is true.

Let $e$ be a truth assignment on $\vec{x}$ such that $\forall\vec{y}\exists\vec{x}\varphi[x/e]$ is true


Let $$K_0:=\{\psi\wedge \neg c'_i\wedge (c''_i\rightarrow w_i)\mid e(c'_i)=0, i=1,\cdots,n\}$$

$$K_1:=\{c'_i\mid e(c'_i)=1, i=1,\cdots,n\}$$

$$K_2:=\{x_i\mid e(x_i)=1, i=1,\cdots, n\}\cup\{\neg x_i\mid e(x_i)=0, i=1,\cdots,n\}$$

$$K:=K_0\cup K_1\cup K_2\cup C_3\cup\{w\}$$

Consider any truth assignment $s$ on $V$.

Suppose $s$ can be extended to satisfy $K$. Say the extenson is $t$. Since $w\in K$, $(c'_1\vee c''_1)\wedge\cdots\wedge(c'_n\vee c'_n)$ be be true under $t$. By formulas in $K_0$, we can see each $w_i$ is true under $t$. That means $s$ make $\beta$ true. Thus, $K\models_V\beta$. \\

Now suppose $s$ satisfies $\beta$. Since $\Phi$ is true, $e*(s\uparrow\vec{y})$ can  be extended to satisfy $\Phi$. Let $t$ be such an extension. Next we show $t$ can be extended to satisies $K$. Since $t$ makes $\varphi$ true, either $c'_i$ is true or $c''_i$ is true. We set each $w_i$ to be true. Then all clauses in $K_0$ are true. Set $w$ to be true. Then clauses in $C_3$ are true. Please note formulas in $K_1\cup K_2$ are already atisfied by $e$. Consequently, $s$ can be extended to satisfy $K$.
Hence, $\beta\models_V K$.\\

Altogether we obtain $K\equiv_V \beta$.
\end{proof}
\end{document}



\section{General Problems}

Configuration and Specification (Sets)\\ \ \\

{\bf Equivalence Configuration Problem}\\

Input: $C=\{\alpha_1,\cdots,\alpha_n\}$ and $\beta$

Query: Whether $\exists K\subseteq C: K\equiv \beta$\\

Let
$$\mbox{IMP}(C,\beta):=\{\alpha\mid \alpha\in C \mbox{ and }\beta\models \alpha\}$$
Then
$$\exists K\subseteq  C(K \text{ is satisfiable and } K\equiv \beta)\Longleftrightarrow\mbox{IMP}(C,\beta) \text{ is satisfiable and IMP}(C,\beta)\models \beta$$

So, equivalence problem is in P$^{\text{NP}[\text{log} n]}$ which is the class of problems solvable in polynomial time with $O(\text{log}n)$ queries to an NP oracle.\\

Next we show hardness. The problem SAT$^n_{\text{odd}}$ is complete for P$^{\text{NP}[\text{log} n]}$. Where SAT$^n_{\text{odd}}$ is the problem of determining whether the number of satisfiable formulas among $n$ given CNF formulas.

We shall construct a reduction from SAT$^n_{\text{odd}}$ to the equivalence configuration problem.

Suppose $F_1,\cdots, F_n$ are 3CNF formulas such that they have pairwisely distinct vaiables. We assume $n$ is even. Otherwise we add formula $p\wedge \neg p$. \\

We also assume each $F_i$ is not tautological (otherwise condider $F_i\wedge p_i$).\\

We assume $F_i$ is over variables $\vec{x}_i$. Pick new varables $\vec{y}_i$ and $\vec{z}_i$. We formulas $F_i(\vec{x}_i), F_i(\vec{y}_i), F_i(\vec{z}_i)$.

$$\psi:=\bigwedge_{i=1}^n \left((F_i(\vec{y}_i)\rightarrow \neg F_i(\vec{x}_i))\wedge (\neg F_i(\vec{y}_i)\wedge \neg F_i(\vec{x}_i)\rightarrow \neg F_i(\vec{z}_i))\right)$$

%$$F_i(\vec{y}_i)\vee F_i(\vec{x}_i)\rightarrow F_i(\vec{z}_i)$$


$$C:=\{\psi\}\cup \{F_1(\vec{y}_1)\vee F_1(\vec{x}_1),\cdots, F_n(\vec{y}_n)\vee F_n(\vec{x}_n)\}$$

$$\beta:=\psi\wedge ((F_1(\vec{y}_1)\vee F_1(\vec{x}_1))\otimes\cdots\otimes (F_n(\vec{y}_n)\vee F_n(\vec{x}_n)))$$


Suppose for some $K\subseteq C$ such that $K\equiv \beta$. Then it must be $\psi\in K$.

We claim that if $F_i$ is satisfiable then $F(\vec{y}_i)\vee F_i(\vec{x}_i)$ must be in $K$.\\


Suppose





\ \\

\section{Restricted Equivalence Configuration}(R-equivalence)}
Introduction R-eqivalence\\
{\bf CNF-Restricted Equivalence}\\
{\bf Instance:}: $C=\{\alpha_1,\cdots,\alpha_n\}$, $\beta$, and a set $V$ of variables\\
{\bf Query:} $\exists K\subseteq C: K\equiv_{V} \beta$?\\



the problem of determining whether $F\equiv_V G$  is in $\Pi_2^P$ (for the complementary problem: we guess a clause $\gamma$ over $V$, check that $G\models \gamma$ but $F\not\models \gamma$, or $G\not\models\gamma$ but $F\models\gamma$).

Thus, R-equivalence configuraton is $\Sigma_3^P$ (guess a $K\subseteq C$, check that it is satisfiable and $K\equiv_V \beta$.) \\


Next we show the hardness.

Consider $\Phi:=\exists \vec{x}\forall\vec{y}\exists \vec{z} \varphi$ where $\varphi$ is a CNF formula. \\

We assume w.l.o.g. that $\varphi$ contains a non-tautological clause over $\vec{z}$ (otherwise, we pick new varaiable $z'$ and consider $\exists\vec{x}\forall\vec{y}\exists\vec{z}\exists z'(\varphi\wedge z'$)).

By this assumption, $\forall\vec{x}\exists\vec{y}\exists\vec{z}\neg \varphi$ is true\\

We assume that each clause contains a positive occurrence of a variable from $\vec{y}\cup\vec{z}$. If otherwise pick a new variable $u$, and consider $\exists \vec{x}\forall\vec{y}\forall u\exists \vec{z}\varphi^u$, where $\varphi^u$ is obtained from $\varphi$ by adding $u$ to clauses containing no positive literal from $\vec{y}\cup\vec{z}$. Clearly, $\Phi$ and the resulting formula has the same truth.

By this assumption, $\forall\vec{x}\exists\vec{y}\exists\vec{z}\phi$ is true. (we will use this later).\\

$\varphi$ can be written as $(c'_1\vee c''_1)\wedge\cdots\wedge (c'_n\vee c''_n)$ in which $c'_i$ is over $\vec{x}$, while $c''_i$ is over $\vec{y}\cup\vec{z}$.

Pick new variables $w_1, \cdots, w_n,w$. Let

Let $$\psi:= \left(\bigwedge_{i=1}^n (c'_i\rightarrow w_i)\right)\wedge
\left(\bigwedge_{i=1}^n(\neg c'_i\wedge \neg c''_i\rightarrow \neg w_i)\right)$$

Let $$C_0:=\{\psi\wedge \neg c'_1\wedge  (c''_1\rightarrow w_1) , \cdots, \psi\wedge\neg c'_n\wedge (c''_n\rightarrow w_n)\}$$

$$C_1=\{c'_1, \cdots, c'_n, \}$$

$$C_2=\{x_1,\neg x_1\cdots, x_n,\neg x_n\}$$

$$C_3=\{(w\rightarrow (c'_1\vee c''_1)\wedge\cdots\wedge (c'_n\vee c''_n))\}$$

Now let $$C:=C_0\cup C_1\cup C_2\cup C_3\cup\{w\}$$

Let $\beta$ be $$ w_1\wedge\cdots\wedge w_n\wedge w$$

Let $V:=\vec{y}\cup\{w_1,\cdots,w_n,\}\cup\{w\}$\\


%Claim: Suppose $K\subseteq C$ satisfiable, $c_i\not \in K$ and $K\not\models c_i$, %then
%$K\cup\{\neg c_i, \psi\wedge (c''_i\rightarrow w_i)\}$ is still satisfiable.


(Lemma: $F\models_V G$ iff for for any truth assignment $t$ on $V$, if it can be extended to a satisfying truth assignment for $F$, it can be extended to an satisfying truth assignment for $G$.
) \\

We shall show $\Phi$ is true if and only if there is satisfiable $K\subseteq C$ s.t. $K\equiv_V \beta$\\



\color{red} From right to left:\color{black}

Given a such $K\equiv_V \beta$.
Clearly, $K\cap C_0$ is non-empty.\\



It must be that $w\in K$.
Let $t$ satisfy $K$.
By our assumption we know $t\uparrow\vec{x}$ can be extended to $s$ which falsifies $\varphi$. Now we assign truth values to $w_i$ according to the tuth values $s(c'_i)$ and $s(c''_i)$ to make formulas $\psi$ and  $(c''_i\rightarrow w_i)$ to be true. There  must be some $w_i$ which is false. Then if we set $w$ to be false, $K$ is still be satisfied by $s$ (because clauses in $C_3$ are satisfied). This contradict the $V$-euivalence.\\

It must be that $C_3\subseteq K$, i.e., $w\rightarrow (c'_1\vee c_1'')\wedge\cdots\wedge (c'_n\vee c''_n)$ is in $K$. Suppose it is not the case. Similar as above, there would be a satisfying truth assigment of $K$ which makes some $w_i$ false, contradicts the $V$-equivalence. \\


Suppose $K\not\models c'_i$. Then $\psi\wedge\neg c'_i \wedge (c''_i\rightarrow w_i)$ must be in $K$.
Otherwise, $K\cup\{\neg c'_i\}$ is consitent,  then there is a satisfying truth assignment $t$ for $K$ such that $t(c'_i)=0, t(w)=1, t(w_i)=0$. (In fact by our assumption, $t$ can be assumed to satisfiy $(c'_1\vee c''_1)\wedge\cdots\wedge (c'_n\vee c''_n)$ because every clause contains positive literal from $\vec{y}\cup\vec{z}$). This contradict the fact that $K\equiv_V\beta$.


Now we can see for each $i$, either $K\models \neg c'_i$ or $K\models c'_i$.
That is, for any two satisfying truth assignments $t_1, t_2$ of $K$, each $c'_i$ has the same truth under $t_1\uparrow \vec{x}$ and $t_2\uparrow\vec{x}$.  \\





Now fix a trut assigment $e$ on $\vec{x}$ which can be extended to a satisfying truth assigment of $K$.

Consider any truth assigment $s$ on $\vec{y}$. Since $s$ can be extended to satisfy $\beta$, it can be extended to a truth assigment $t$ which satisfies $K$. Please note we can assmue that $t\uparrow\vec{x}$ is $e$.
Since $w$ and $w\rightarrow (c'_1\vee c''_1)\wedge\cdots\wedge (c'_n\vee c''_n)$ are in $K$. We can see $t$ satisfy $\varphi$.

Consequently, for any truth assigment $s$ on $\vec{y}$, $e*s$ can be extended to a satisfying truth assigment of $\varphi$. Thus, $\Phi$ is true.


\ \\

{\color{red} From left to right.\color{black}


Suppose $\Phi=\exists\vec{x}\forall\vec{y}\exists\vec{z}\varphi$ is true.

Let $e$ be a truth assignment on $\vec{x}$ such that $\forall\vec{y}\exists\vec{x}\varphi[x/e]$ is true


Let $$K_0:=\{\psi\wedge \neg c'_i\wedge (c''_i\rightarrow w_i)\mid e(c'_i)=0, i=1,\cdots,n\}$$

$$K_1:=\{c'_i\mid e(c'_i)=1, i=1,\cdots,n\}$$

$$K_2:=\{x_i\mid e(x_i)=1, i=1,\cdots, n\}\cup\{\neg x_i\mid e(x_i)=0, i=1,\cdots,n\}$$

$$K:=K_0\cup K_1\cup K_2\cup C_3\cup\{w\}$$


Consider any truth assignment $s$ on $V$.

Suppose $s$ can be extended to satisfy $K$. Say the extenson is $t$. Since $w\in K$, $(c'_1\vee c''_1)\wedge\cdots\wedge(c'_n\vee c'_n)$ be be true under $t$. By formulas in $K_0$, we can see each $w_i$ is true under $t$. That means $s$ make $\beta$ true. Thus, $K\models_V\beta$. \\

Now suppose $s$ satisfies $\beta$. Since $\Phi$ is true, $e*(s\uparrow\vec{y})$ can  be extended to satisfy $\Phi$. Let $t$ be such an extension. Next we show $t$ can be extended to satisies $K$. Since $t$ makes $\varphi$ true, either $c'_i$ is true or $c''_i$ is true. We set each $w_i$ to be true. Then all clauses in $K_0$ are true. Set $w$ to be true. Then clauses in $C_3$ are true. Please note formulas in $K_1\cup K_2$ are already atisfied by $e$. Consequently, $s$ can be extended to satisfy $K$.
Hence, $\beta\models_V K$.\\

Altogether we obtain $K\equiv_V \beta$.








\ \ \\

\section{Implication Problem 1}

Let $X$ be qa classs of propositional formulas, for example CNF or HORN. Then we define
{\bf Problem X-Implication}\\
{\bf Instance}: A set of components $C=\{\alpha_1,\cdots,\alpha_n\}$ and the target formula $\beta$, where the formulas are in X.\\
{\bf Query}: Does there exist $K \subseteq C: K$ is satisfiable and $K\models \beta$ ?\\

\begin{lemma}
The CNF-Implication Problem is $\Sigma^P_2$--complete.
\end{lemma}

\begin{proof}
Clearly, the problem is in $\Sigma_2^P$.  Now it remains to show the hardness:\\
Let $\Phi:= \exists x_1, \ldots, x_n \forall y_1, \ldots, y_m \varphi$ be a closed formula, where
$\varphi$ is a propositional DNF-formula. For a new variable $z$ we define the set of components\\
$C= \{ \varphi \rightarrow z, x_1, \neg x_1, \ldots, x_n, \neg x_n\}$ and the target formula $\beta=z$. Please note, that v$\varphi \rightarrow z$ can easily transformed into an equivalent CNF-formula. Then it holds:\\
$\Phi$ is true if and only if $\exists K \subseteq C: K \in SAT$ and $K\models z$.\\
From left to right:\\
Suppose $\Phi$ is true. Then there is a partial truth assignment $v(x_1) \epsilon_1, \ldots v(x_n)= \epsilon_n$, such
that $\forall \varphi(x_1/\epsilon_1, \ldots, x_n/\epsilon_n)$ is true. The subset of components
$K= \{ \varphi \rightarrow z, x_1^{\epsilon_1}, \ldots, x_n^{\epsilon_n}\}$ is satisfiable. Assume that $K\not \models z$.
Then there is a truth assignment $v$ with $v(z)=0$, but $v(K)=1$. Then we obtain $v(\varphi[x_1/\epsilon_1, \ldots, x_n/ \epsilon_n) \rightarrow z)=1$ and therefore $v(\varphi[x_1/\epsilon_1, \ldots, x_n/ \epsilon_n)=0$ as a contradiction.\\

From right to left:


\end{proof}


{\bf Problem 1-CNF-Implication}\\
{\bf Instance}: A set of components $C=\{\alpha_1,\cdots,\alpha_n\}$ and the target formula $\beta$, where the formulas are in 1-CNF.\\
{\bf Query}: Does there exist $K \subseteq C: K$ is satisfiable and $K\models \beta$ ?\\

\begin{lemma}
The 1-CNF Implication problem is NP-complete.
\end{lemma}
\begin{proof}
In NP: guess $K \subseteq C$, test for satisfiability is linear, $\models$ too.\\

NP-hardness: sets $M_i :=\{p_{i_1}, p_{i_2}\}$, $C=\{M_1, \ldots, M_k\}$, over $\beta := p_1\wedge \ldots\wedge p_n$\\
exclude: not ($M_i$ and $M_j$ for some $i$ and $j$.\\
\begin{center}
$\alpha_i = p_{i_1} \wedge p_{i_2} \wedge a(i) \wedge \bigwedge_{\{i,j\} \in E}\neg a(j)$\\
$C = \{\alpha_1, \ldots, \alpha_k\}$\\
$\beta= p_1 \wedge \ldots \wedge p_n$
\end{center}

\end{proof}





If the formulas are definite Horn formulas, then the problem is quite simple. There exists a solution, if for the complete set components $C$ and the target formula $\beta$ it holds $C \models \beta$. When we modify the query to minimal $K \subseteq C$ or to a subset $K$ with at most $k$ components the computational complexity jumps to NP-complete.
That can be seen as follows:\\
{\bf Instance:} $C=\{\alpha_1, \ldots, \alpha_n\}$, target formula $\beta$ are DHorn formulas, integer $k$\\
\{\bf Query:} $\exists K \subseteq C: |K| \leq k, K \models \beta$?\\
We can establish a reduction to the NP-complete problem called Minimal Input Set \cite{GaJo79}.
\begin{center}
({\bf MI}) Instance: An acyclic definite Horn formula $\alpha$, a variable $a$ occuring positively in $\alpha$, and $k \geq 0$.\\
Query: Does there exists a minimal input set $S$ for $\alpha$ and $a$ such that $|S| \leq k$?
\end{center}
An input set for $\alpha$ and $a$ is a set $I$ of variables which only occur as negative literals in $\alpha$ and
for which $I \wedge \alpha \models a$.\\

Then our reduction is as follows:\\
We define $C= \{\alpha\} \cup I$, target formula $a$, and let $k' := k+1$. Then there is a $K \subseteq C$ with
$|K | \leq k'$ if and only if {\bf MI} has a solution.


There is a big difference to Horn formulas, because a set of Horn formulas can be unsatisfiable

{\bf Problem Horn-Implication}\\
{\bf Instance}: A set of components $C=\{\alpha_1,\cdots,\alpha_n\}$ and the target formula $\beta$, where the formulas are Horn formulas.\\
{\bf Query}: Does there exist $K \subseteq C: K$ is satisfiable and $K\models \beta$ ?\\

\begin{lemma}
The Horn-Implication and the 2-CNF-Implication Problem are NP--complete.
\end{lemma}

\begin{proof}
For Horn formulas $\phi$ with at least one negative clause the following problem is NP-complete:\\
$\exists x_1 \in \{a_1, a_2\} \ldots \exists x_n \in \{a_n, b_n\}: (\phi \wedge \bigcup_{1 \leq i \leq n} x_i \in \overline{SAT})$\\
For a new variable $p$ we substitute any negative clause $\phi_j$ in $\phi$ with $(\phi_j \vee p)$. The resulting formula
is denoted as $\phi(p)$. Then we obtain \\
$\exists x_1 \in \{a_1, a_2\} \ldots \exists x_n \in \{a_n, b_n\}: (\phi \wedge \bigcup_{1 \leq i \leq n} x_i \in \overline{SAT})$\\
if and only if\\
$\exists x_1 \in \{a_1, a_2\} \ldots \exists x_n \in \{a_n, b_n\}: (\phi(p) \wedge \bigcup_{1 \leq i \leq n} x_i \models p$\\

Next we define the components:
$C :=\{\phi(p), \sigma_1, \pi_1, \ldots, \sigma_n, \pi_n\}$, where $\sigma_i= a_i \wedge \neg b_i$ and
$\pi_i= \neg a_i \wedge b_i$. Then it holds\\
$\exists K \subseteq C: (K \in SAT$ and $K \models p$) if and only if $D_2$ holds.


\end{proof}


Note: In all problems $K$ is demanded satisfiable.

$$\alpha \equiv_V \beta \Longleftrightarrow \forall \gamma \mbox{ over } V, (\alpha\models \gamma \Leftrightarrow \beta\models \gamma) $$

$$\alpha \models_V \beta \Longleftrightarrow \forall \gamma \mbox{ over } V, (\beta\models \gamma \Longrightarrow \alpha\models \gamma) $$


%%%%%%%






%%%%%%%%%%%%%%%%%%%%%%%%%%%


\section{Resticted to DHORN}

$C$: set of DHORN, $\beta$: DHORN \\

{\bf Equivalence problem is in PTIME} \\

Idea: just checked IMP$(C,\beta)\equiv \beta$.



%%%

\ \\

{\bf Implication problem is trivial} \\

The existence of $K$ is equivalent to $C\models\beta$\\

\ \\


{\bf The $V$-equivalence problem}\\

%IMP$(C,\beta,V):=\{\alpha\in C\mid \beta\models_V \alpha\}$\\

%The existence of $K\equiv_V \beta$ is equivalent to IMP$(C,\beta, V)\equiv_V\beta$.\\

we guess it seems $\Sigma_2^P$-complete.

See Hans's book page 251


\section{Restricted to HORN}


Equivalence Problem: same as the DHORN case (PTIME).\\

Desired $K$ exists iff IMP$(C,\beta)$ is satisfiable and IMP($C,\beta)\models \beta$\\

\ \\

V-Equivalence Configuration Problem :  $\Sigma_2^P$-complete\\


$$\Phi:=\exists y_1,\cdots, y_k \forall x_1,\cdots, x_m \left(c_1\vee\cdots\vee c_n \right)$$
%
Where $c_i$ is a conjunction of literals.\\

for each variable $z$, introduce a new variable $\pi(\neg z)$. Let $\pi(z)=z$.\\

introduce $U$.\\

Let $$\psi_0:=\left(\bigwedge_{j=1}^m (\neg \pi(\neg x_j)\vee \neg x_j)\right), \ \psi_1:= \left(\bigwedge_{i=1}^k(\neg \pi(\neg y_i)\vee \neg y_i)\right)$$

$$\theta:=\left(\bigvee_{i=1}^m (\neg x_i\wedge \neg\pi(\neg x_i))\right)$$

%$$\theta_i:=\left(\bigvee_{x\not\in c_i, \neg x\not\in c_i}(\neg x\wedge\neg\pi(\neg %x))\right)$$

Please note that $\theta$ is not HORN  when $n>1$. Forturnately, by using Tseiting algorithm, one can transform  $\theta$ to an equivalent HORN formula when restricted to old variables. \\



For a conjunction $c=L_1\wedge\cdots\wedge L_s$, we write
$\pi(c):=\pi(L_1)\wedge\cdots\wedge \pi(L_s)$


$$\begin{array}{lcl}C_0&:=&\{\psi_0\wedge\psi_1 \wedge\left( \bigwedge_{i=1}^n(\pi(c_i)\rightarrow U\vee \theta)\right)\}\\

C&:=&C_0\cup \{y_i,\pi(\neg y_i)\mid i=1,\cdots,k\}\\
\beta&:= &(U\vee\theta)\wedge \psi_0 \\
V&:=&\{U\}\cup\{x_1,\cdots,x_m,\pi(\neg x_1),\cdots, \pi(\neg x_m)\}
\end{array}$$

We shall show

$$(\exists K\subseteq C \text{ such that } K\equiv_V \beta) \text{ if and only if $\Phi$ is true}.$$

\ \\

\color{red}
!!!! I suddenly find that $U\vee \theta $ may not be HORN. Forturnately we can change $U$ to $\neg U$ in $C$ and $\beta$. Then the following proof should be changed accordingly
\color{black}\\

Suppose $\Phi$ is true. There is truth assignment
$e$ on $\{y_1,\cdots, y_m\}$ such that $\forall x_1,\cdots,x_m\varphi[\vec{y}/e]$ is true, where $\vec{y}=y_1\cdots,y_k$. Define
%
$$K=C_0\cup \{y_i\mid t(y_i)=1, 1\leq i\leq k\}\cup \{\pi(\neg y_i)\mid t(y_i)=0, 1\leq i\leq k\}$$

For any satisfying truth assignment $t$ for $K$, if it makes $\theta$ true then $\beta$ is true under $t$. Suppose $t$ makes $\theta$ false, then $t$ corresponds a truth assignment on $x_1,\cdots, x_n$ accoding the truth of $x_i$ and $\pi(\neg x_i)$. Since $\Phi$ is true, some  $\pi(c_i)$ must be true under $t$, then $T(U)=1$. Therefore, $t(\beta)=1$.

Suppose $s$ is a satisfying truth assignment for $\beta$. Then $s*e$ satisfies $\psi_0\wedge \psi_1$. If $s$ makes $\theta$ true, then $s*e$ satisfies $K$. Suppose $s(\theta)=0$. Then $s(U)=1$, hence $K$ is still satisfied by $s*e$.

Altogether, we have $K\equiv_V\beta$.\\


For the inverse direction, suppose there is satisfiable $K\subseteq C$ such that $K\equiv_V\beta$.


Clearly, the formula in $C_0$ must be in $K$. Then by formula $\psi_1$, either $y_i$ or  $\pi(\neg y_i)$ is not in $K$.

Pick a truth assignment $e$ on $\{y_1,\cdots, y_k\}$ such that if $y_i\in K$ then $e(y_i)=1$, and $e(y_i)=0$  if else.

We need to show $\forall x_1,\cdots,x_m(c_1\vee\cdots\vee c_n)$ is true.

Consider any truth assigment $s$ on $\{x_1,\cdots, x_m\}$. Let $s'$ be the assignment defined by $s'(\pi(\neg x_i)=1$ iff $s(x_i)=0$. Then $e'*s'$ satisfies $\psi_0\wedge\psi_1$, where $e'$ is obtained from $e$ is the same way as $s'$.
We claim that, $(e'*s')$ makes $\pi(c_i)$ true for some $i$. Otherwise, we could set $U$ to be false,  and get a truth assignment satisfying $K$ but bot satisfying $\beta$ (please note that $s'(\theta)=0$), contradict the $V$-quivalence.

Consequently, $\Phi$ is true.

\ \\

\section{Implication Problem}

NP-complete

Given a 3CNF  $F$

$$\bigwedge_{i=1}^m (L_{i,1}\vee L_{i,2}\vee L_{i,3}) \ \mbox{ over } x_1,\cdots, x_n$$

For each $i=1,\cdots,m$, pick a new variable $z_i$. For each $j=1,\cdots, n$ we pick a new variable $\pi(\neg x_j)$. For convenience, we also write $x_j$ as $\pi(x_j)$.

Define $C$

$$\begin{array}{ll}C:=&
\bigcup_{i=1}^m\{\pi(L_{i,1})\rightarrow z_i, \pi(L_{i,2})\rightarrow z_i, \pi(L_{i,3})\rightarrow z_i\}\cup\\ & \bigcup_{j=1}^n\{\rightarrow x_j, \rightarrow \pi(\neg x_j\} \cup\\ &
\{z_1\wedge\cdots\wedge z_m\rightarrow z\}\end{array}$$

Define

$$\beta:=z\wedge\bigwedge (\neg \pi(\neg x_j)\vee \neg x_j)$$


Impliation problem iff $F$ is satisfiable \\


\ \\


Specification Problem\\

Given a partial configuration $K$, demand $\beta$, a set of variables $V,W$

Looking for $\sigma$ over $W$ such that

\begin{enumerate}
\item $K\wedge \sigma(W)\equiv \beta$

\item $K\wedge\sigma(W) \equiv_V\beta$

\item $K\wedge\sigma(W)\models \beta$
\end{enumerate}


Query Learning

black box $\alpha$

equivalence query. Guess a $\beta$ ask whether $\alpha\equiv \beta$. If the answer is no, output a truth assignment satisfying $\alpha$ and $\neg \beta$ or satisfying $\neg\alpha$ and $\beta$.\\

membership query

guess a truth assignment $v$ answer $v(\alpha)=0$ or 1. \\



%%%%%%%%%






\section{Configuration with constraints}
For various applications not only the set of components and the target formula are given, but additionally
global dependencies - called global knowledge - supports the configuration process or leads to
some constraints. For example, we want mutually exclude that two components say $\alpha_1$ and $\alpha_2$
occur in the solution.
Such global knowledge may lead to relatively short running times for the configuration process.

{\bf Configuration-Sets-GK}\\
{\bf Instance}: $p_1, \ldots, p_n$, and $M = \{ M_1, \ldots, M_t\}$ for $M_i =\{p_{i_1}, p_{i_2\}$ for $1 \leq i \leq t$.\\
{\bf Constraints}: $E \subseteq \{1, \ldots, t\}^2$\\
{\bf Query}: Does there exist $K \subseteq M: \bigcup_{L \in K} L =\{p_1, \ldots, p_n\}$ and
($M_j \not \in K$ or $M_k \not \in K$) for all pairs $(j,k) \in E$.\\

\begin{theorem}
The Configuration-Sets-GK Problem is NP-complete.
\end{theorem}
Transformation into log-problem.\\

$C=\{\alpha_1,\cdots, \alpha_n\}, \beta$,
$D$ is set of formulas over $A_1,\cdots, A_n$ which are propositional atoms.
Whether there is $K\subseteq C$ such that
\begin{itemize}
\item $K$ is satisfiable,
\item $K\equiv\beta$, and
\item the truth assignment $v_K$ satisfies $D$, where $v_K(A_i)=\left\{\begin{array}{ll}
1 & \text{if } \alpha_i\in K\\ 0 & \text{if }\alpha_i\not\in K\end{array}\right.$

\end{itemize}

Duppose $D$, $\beta$ are DHORN, $D$ is 2HORN, then NP-complete.\\

%2HORN formula $H$, and $t$ is a conjunction of propositional atoms, $V$ a set variables with distint atoms from $t$.

%Determine whether there is a set $s\subseteq V$ such that $H$ is consistent and $s\wedge H\models t$.

%This problem is NP-complete.

Let $F$ be an arrbitrary 3CNF formula $c_1\wedge \cdots\wedge c_n$ with $c_i:=l_{i,1}\vee l_{i,2}\vee l_{i,3}$  over $\{x_1,\cdots, x_m\}$


pick new variable $\pi(\neg x_i)$ for $\neg x_i$. $\pi(x_i)$ is $x_i$.

introduce new variables $w_1, \cdots, w_n$. Let

$$\begin{array}{l}H_0=\{\pi(l_{i,j})\rightarrow w_i\mid 1\leq i\leq n, j=1,2,3\}\\
D:=\{\neg x_1\vee \neg \pi(\neg x_1), \cdots, \neg x_m\vee \neg \pi(\neg x_m)\}\\
H:=H_0\cup D\end{array}$$,

Let $\beta:=w_1\wedge\cdots\wedge w_n$. $V=\{x_1, \pi(\neg x_1),\cdots, x_m,\neg x_m\}$.

Clearly $F$ is satisfiable iff there is $s\subseteq V$ such that $s\wedge H\models \beta$. \\


Next we transform $H$ to the constraint configuration problem for DHORN.



For each literl $L$ over $\{x_1,\cdots, x_m\}$, define $$\alpha_i(\pi(L)):=\{w_j\mid \pi(L)\rightarrow w_j\in H_0\}$$

%For each $\pi(x_i)$, define $$\alpha_i^-=\{w_j\mid \pi(\neg x_i)\rightarrow w_j\}$$

Let $$K:=\{\alpha(\pi(L))\mid L\in\{x_1, \neg x_1,\cdots, x_m\neg x_m\}\}$$
%%
Now we can consider $\pi(L)$ the name of $\alpha(\pi(L))$. So, let
$$D:=\{\neg x_1\vee \neg\pi(\neg x_1),\cdots, \neg x_m\vee\neg \pi(\neg x_m)\}$$

%
Clearly $K\equiv \beta$.

Clearly, there is $s\subseteq V$ such that $v\wedge H\models \beta$ if and only if

there is $s\subseteq V$ such that $s$ satisfies $D$ when $s$ is considered as a truth assigment and $\{\alpha(L)\mid L\in s\}\equiv \beta$.

Thus the constraint configuration problem for this case it NP-complete. \\



However, if $C:=\{\alpha_1,\cdots, \alpha_n\}, \beta$ and $D$ are all DHORN, then it is poly-time solvable.

IMP$(C, \beta)$. Suppose it is satisfiable. (otherwsie retun no)

Please note that for any $\alpha\in C-\text{IMP}(C,\beta)$, $\alpha\wedge \text{IMP} \not\equiv \beta$


Let $D':=D\cup\{\neg A_i\mid \alpha_i\not\in\text{IMP}(C,\beta), \}$.

If $D'$ is not satisfiable, then return no.

Suppose $D'$ is sat.

Check that IMP$(C,\beta)\equiv\beta$. if not return no. Suppose it is the case.

Let IMP$_1:=\text{IMP}()-\{\alpha_i\mid D'\models \neg A_i\}$

Check that IMP$_1$ equivalent to $\beta$. If it is the case then return yes, return no else.







\end{document} 