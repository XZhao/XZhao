\documentclass[12pt]{article}
\usepackage{graphicx}
\usepackage{makeidx}
\usepackage{amsmath}
\usepackage{amsfonts}
\usepackage{color}
\usepackage[all]{xy}



%%%%%%%%%%%%%%%%%%%%%%%%%%%%%%%%%%%%%%%%%%%%%%%%%%%%%%%%%%%%
\long\def\remove#1{}
\newcommand{\dom}{\mbox{dom}}

\newcommand{\vp}{\varphi}


\newcommand{\coNP}{\mbox{coNP}}

\newcommand{\NP}{\mbox{NP}}
\newcommand{\DP}{\mbox {D}^P}

\newcommand{\LeKl}{\mbox{$\sqcup$ }}
\newcommand{\card}[1]{\mbox{\#}(#1)}

\newcommand{\sss}{\mbox{\bf S}}
\newcommand{\jjj}{\mbox{\bf J}}

\newcommand{\QR}[1]{\mbox{$\ \mid\!\!\!\frac{#1}{\
    \stackrel{\mbox{\scriptsize\it Q-Res}}{\ }\ }\ $}}
\newcommand{\QUR}[1]{\mbox{$\ \mid\!\!\!\frac{#1}{\
    \stackrel{\mbox{\scriptsize\it Q-Pos-Unit-Res}}{\ }\ }\ $}}
%\baselineskip 0.2in


\newcommand{\pbox}{\hbox to 6pt{\leaders\hrule width 6pt height 6pt\hfill}}

\newtheorem{definition}{Definition}
\newtheorem{theorem}{Theorem}
\newtheorem{lemma}{Lemma}
\newtheorem{corollary}{Corollary}
\newtheorem{proposition}{Proposition}
\newenvironment{proof}{\parindent=0pt{\bf Proof: }}{
   \hspace*{\fill}\hbox to 6pt{\leaders\hrule width 6pt height 6pt\hfill}\par}


\pagestyle{plain}

\begin{document}



%%%%%%%

\title{
}

\author{
Xishun Zhao \footnote{Corresponding author. Tel: 0086-20-84114036,
Fax:0086-20-84110298.}
%\thanks{This research was partially supported by the NSFC project
%under grant number: 60970040 and a MOE project
%under grant number: 05JJD72040122. }\\
%Institute of Logic and Cognition,
%\\ Sun Yat-sen
%University\\ 510275 Guangzhou, (P.R. China)\\
%{Email: hsszxs@mail.sysu.edu.cn}
}


\maketitle

\begin{abstract}

\end{abstract}
%%%%%%%%%%%%%%%%%%%

\begin{theorem}
$\Phi:=Q(\varphi+f)$, $f=u_1\vee \cdots\vee u_s\vee f'$, 
each $u_i$ is a universal literal, $u_i>l$ for all existential literal $l$ of $f$.  Then 

$\Phi$ is true iff both $Q\varphi$ and $Q'(\varphi[u_1=0,\cdots,u_s=0]+f'$ are true. 
\end{theorem}

\begin{proof}

$Q$ is $\forall Y\exists X$. 

The direction $\Longrightarrow$ is clearly valid. 



The inverse direction $\Longleftarrow$ is also nard to see.\\

$Q$ is $\exists xQ*$ then neither $x$ nor $\neg x$ occurs in $f$. 

Suppose $\Phi$ is true. and $Q^*(\varphi[x=\epsilon]+f)$ is true. 

By the  induction hypothesis $Q^*(varphi[x=\epsilon])$ is true and $Q'(\varphi[x=\epsilon][u_1=0,\cdots, u_s=0]+f')$ is true. 

By the former, we have $\exists xQ^*varphi$ is true. 

By the latter, 

%%%%%%%%%%%%%%%%

For the inverse. Suppose $\exists xQ^*\varphi$ is true and $\exists xQ^*( +f')$ is true. 




%%% something wrong



\end{proof}


%%%%%%%%%%%%%%%
QCNF$(k)$ is the class of QCNF formulas with maximum defciency $k$. \\

QSAT$(k)$ is the class of all true formulas in QCNF$(k)$. 

Suppose for $k\geq 1$, QSAT$(k)$ is solvable in polynomial time $p(n)$, where $n$ is the length of the formula. \\

Next we show QSAT$(k+1)$ is also solvable in polynomial time. 

Let $\Phi:=Q\varphi$ with maximum deficiency $K+1$.\\

Case 1. $Q$ is $\forall y Q'\varphi$. 

Case 1.1. $y$ occurs in $\varphi$ positively and negatively 

      Check the truth of $Q'\varphi[y=1]$ and $Q'\varphi[y=0]$.  
      
      If both are true then return true, else return false. \\
      
Case 1.2. $y$ occurs in $\varphi$ only positively

      Consider $Q'\varphi[y=0]$\\
      
Case 1.3. $y$ occurs only negatively 

      Consider $Q'\varphi[y=1]$\\
      

Case 2. $Q$ is $\exists XQ'\varphi$, where $Q'$ begins with universal quantifiers. 

pick a clause $\varphi\in \varphi$ such that var$(f)\cap X$ is non empty.

Let $L_1, \cdots, L_m$ be literals of $f$ such that var$(L_i)\in X$.
%for generality we assume $L_i$ occurs in $\varphi-\{f\}$. 

$\varphi'=\varphi-\{f\}$ %has fewer deficiency

For each literal $L_i$ $\{$ let $X_i=X-\{\text{var}(L_i)\}$,    

If one of $\exists X_i \varphi'[L_i=1]$ is true then return true. $\}$

else $\{$ Let $X*=X-\{\text{var}(L_1),\cdots, \text{var}(L_m)\}$ $\varphi*=\varphi[L_1=0,\cdots,L_m=0]$, and consider

$\exists X^*Q'\varphi^* $      
      
      
  %%%%%%%%%%%%%%%%%%%%%
  \   \\
  
  
$\Phi=\exists X Q(\psi+f)$ such that var$(f)\cap X$ is empty. Let $u_1, \cdots, u_t$ are all universal literals larger than all existential literals in $f$. Let $f'$ obtained from $f$ by deleting all $u_i$. Then 

$\Phi$ is true iff either $\exists XQ\psi$ and $\exists X Q(\psi+f')$ are true.     
      



%\end{document}

%%%%%%%%%%%%%%%%
Given $f, g$, and $H$ such that $f+H\in$ MU(1), $g+H\in$ MU(1). Then 

$$\forall h\in H, \ (H-h)+\{f,g\} \text{ is satisfiable}$$

Proof. For $n=1$, it is clearly true. 

Then by induction and the disjoint splitting propery of MU(1) formulas \\



Suppose $\Phi:=Q\varphi$ with $\varphi=H+\{f,g\}$ with the above property.
Then (conjecture)
%
$\Phi$ is true iff both $Q(H+f)$ and $Q(H+g)$ are ture\\


Need the following property. 

$F\in $MU(1). $f, g \in F$, a path from $f$ to $g$ 

$f_1=f, \cdots, f_n=g$ and $L_1, \cdots, L_n$ such that 

$L_1\in f_1$, $\neg L_i, L_{i+1}$ are in $f_{i+1}$ for $i=1, \cdots n-1$, $\neg L_n \in f_n$. \\


Claim: Let $f,g\in F$, $\pi_1, \pi_2$ are two paths from $f$ to say to $g_1$ and $g_2$. Then the two path are compatable, that is, they do not contain complementary literals. 

Proof. For $n=1$ clearly. For $n>1$ by using disjoint splitting. 
\\

Any two path must compatible\\

{\bf So, in this case, QMU(2) is solvable in polynomial time.}


Suppose $F$ is lean $d(F)=2$. 

Case 1. $F-\{f\}\in$ MU(1), and for all $h\in F-\{f\}$, $F-\{h\}$ is satisfiable.

Case 2. $F-\{f\}\in$ MU(1), and there is $g\in F-f$, $F-g$ is in MU(1) (then this case)

Case3. $F'\subseteq F\in $MU(1) such that $F-F'$ has more than one clauses. 

Case 4. $F\in$ MU(2)

%%%%%%%%%%%%%%%%%%%%%%%%%%%%%%%%%%

\ \ \\ \ \\
Suppose $F=H+f$, $G=H=g$ both are in MU(1) (then $H+f\vee g$ is also in MU(1))
%
Then 


\begin{enumerate}

\item $f^{-x}\in F_x$ iff $g^{-x}\in G_x$

\item $f^{-\neg x}\in F_{\neg x}$ iff $g^{-\neg x}\in G_{\neg x}$

\item $f\in F[x=0]-F_x$ iff $g\in G[x=0]-G_x$

\item $f\in F[n=1]-F_{\neg x}$ iff $g\in G[x=1]-G_{\neg x}$ 

\item In case 3, $(F[x=0]-F_x)\wedge g$ is satisfiable on variables not in $F_x$.

\item In case 4, similarly. 

\end{enumerate}



$\Phi=Q\varphi$, $\varphi=f+\theta + g$


There must be a universal variable $y$, and a partial model 

%%%%%%%%%


\begin{lemma}
$\exists x\forall y\exists z \phi$. 
$M$ is a model of $\Phi$. 

deleting from $\phi[z/M_z]$ all tautological clauses, then deleting all universal part. $M_x$ satisfies the resulting formula (over $x$) is satisfiable. 
\end{lemma}

Examples

$$\color{red} (x_1\vee x_2) \color{black} \wedge (\neg x_1\vee x_2) \wedge (\neg x_2\vee x_3)\wedge (\neg x_2\vee \neg x_3)\wedge \color{red} (x_1)\color{black}$$


$$\color{black} (x_1\vee x_2) \color{black} \wedge (\neg x_1\vee x_2) \wedge (\neg x_2\vee x_3)\wedge (\neg x_2\vee \neg x_3)\wedge \color{red} (x_2\vee x_3)\color{black}$$


%%% other new idea

$\exists X\forall Y\exists \cdots \varphi$ 

for each $y\in Y$, consider all clauses containing $y$ and all clauses containing $\neg y$. For each truth assignment on $X$, the clauses with $y$ or the other becomes a autarky. whether poly-time computable?   


\section{General Structure of LEAN(2)}

Suppose $F\in\text{MLEAN}(2)$ unsatisfiable, but $F\not\in \text{MU}(2)$. 

Then $F$ must contain a MU(1) subformula. So, let $G\subseteq F$ be a subformula in MU(1). Let $\theta=F-G$. 

Plaese note that $F$ is matching lean, the formula $\theta'$ obtained from $\theta$ by omitting variables occuring in $G$ must be mlean and has deficiency 1.  

{\bf Case 1.} For any clause $g\in G$  such that $(G-\{g\})+\theta$ is satisfiable. That is, $G$ is the only MU(1) subformula of $F$. 

We suppose it is not case 1. 

   
{\bf Case 2.} For some $g\in G$ such that $(G-\{g\})+\theta$ is unsatisfiable.
That is there is $G'\subseteq F$ such that $G'\not=G$ and $G'\in \text{MU}(1)$. 

Because $G'=G'\cap G+G'\cap \theta$ and $G'$ has deficiency 1, it must be that $\theta\subseteq G'$. 

Let $H=G\cap G'$, $\eta =G'-H$. That is, $F$ can be written as follows

$$F=\theta+H+\eta\ \text{ with }\theta+H\in \text{MU}(1) \text{ and } \eta+H \in \text{MU}(1)$$ 

Let $\theta^-$ (resp. $\eta^-$) be the formula obtained from $\theta$ (resp. $\eta$) by omitting occurrences of variables of $H$. Then both $\theta^-$ and $\eta^-$ are in MU(1) and have distinct variables. 

The following is an example of such formulas \\


Let $\theta^+$ is the clause obtained from $\theta$ by iteratively applying sDP reduction on variables in $\theta^-$. Likewise for $\eta^+$. Then we have 

$$\{\theta^+\}+H,\ \{\eta^+\}+H \text{ and } \{\theta^+\vee \eta^+\}+H \text{ are all in MU(1)}$$ 

%%%%%%%%

We consider the first case. That $F$ contains only MU(1) subformula.

Agian try and try


\end{document}