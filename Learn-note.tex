\documentclass[12pt]{article}
\usepackage{graphicx}
\usepackage{makeidx}
\usepackage{amsmath}
\usepackage{amsfonts}
\usepackage{color}
\usepackage[all]{xy}

%%%%%%%%%%%%%%%%%%%%%%%%%%%%%%%%%%%%%%%%%%%%%%%%%%%%%%%%%%%%
\long\def\remove#1{}
\newcommand{\dom}{\mbox{dom}}

\newcommand{\vp}{\varphi}


\newcommand{\coNP}{\mbox{coNP}}

\newcommand{\NP}{\mbox{NP}}
\newcommand{\DP}{\mbox {D}^P}

\newcommand{\LeKl}{\mbox{$\sqcup$ }}
\newcommand{\card}[1]{\mbox{\#}(#1)}

\newcommand{\sss}{\mbox{\bf S}}
\newcommand{\jjj}{\mbox{\bf J}}

\newcommand{\QR}[1]{\mbox{$\ \mid\!\!\!\frac{#1}{\
    \stackrel{\mbox{\scriptsize\it Q-Res}}{\ }\ }\ $}}
\newcommand{\QUR}[1]{\mbox{$\ \mid\!\!\!\frac{#1}{\
    \stackrel{\mbox{\scriptsize\it Q-Pos-Unit-Res}}{\ }\ }\ $}}
%\baselineskip 0.2in


\newcommand{\pbox}{\hbox to 6pt{\leaders\hrule width 6pt height 6pt\hfill}}

\newtheorem{definition}{Definition}
\newtheorem{theorem}{Theorem}
\newtheorem{lemma}{Lemma}
\newtheorem{corollary}{Corollary}
\newtheorem{proposition}{Proposition}
\newenvironment{proof}{\parindent=0pt{\bf Proof: }}{
   \hspace*{\fill}\hbox to 6pt{\leaders\hrule width 6pt height 6pt\hfill}\par}


\pagestyle{plain}

\begin{document}


\title{
}

\author{
Xishun Zhao \footnote{Corresponding author. Tel: 0086-20-84114036,
Fax:0086-20-84110298.}
%\thanks{This research was partially supported by the NSFC project
%under grant number: 60970040 and a MOE project
%under grant number: 05JJD72040122. }\\
%Institute of Logic and Cognition,
%\\ Sun Yat-sen
%University\\ 510275 Guangzhou, (P.R. China)\\
%{Email: hsszxs@mail.sysu.edu.cn}
}


%\authorrunning{X. Zhao, H. Kleine B\"{u}ning}



\maketitle

\begin{abstract}

\end{abstract}
%%%%%%%%%%%%%%%%%%%
%note on queries and concept learning author dana angluin

An universe $U$, a hypothesis space ${\cal S}=\{L_1, \cdots, \}$ of susbets of $U$. $S$ can be infinite. An unknown susbet $L_{*}$. 
 After some queries, learn $L_{*}$. 
 
Quesries include:

Memebership query:

Equivalence query 

Subset query

Supserset query

Disjointness query

Exhaustiveness query\\

When an input of a query is a subset of $U$, then it must be in ${\cal S}$.


%%%% example

poker hand


a card: a number and a suit(S,H,D,C))

A hand is a set of five cards, without any order

A pair of hands is an ordered pair of hands with no card in common

The universe $U$ is the set of all pairs of hands. \\

%%%%%%%%%%%%%%%%

Exact identification: after some queries to find an index $i$ such that the target notion is exactly $L_i$ in ${\cal S}$.\\


Probablistic identification (by L.G.Valiant 1988).

a disributuion $D$ on $U$. Pr$(x)$ is the probability of element $x$ wrt $D$

Sampling oracle EX( ) which has no input. When EX( ) is called it returns an element $x$ with an identification of whether $x$ is in the target set

difference of two sets $L_1, L_2$: $d(L_1, L_2):=\Sigma_{x\in L_1\otimes L_2} Pr(x)$


two parameters $\epsilon$ accuracy, $\delta$ confidence.

probably approximately correctly(pac) identification: always halts and output an index $i$ such that 
$Pr(d(L_*, L_i)\leq \epsilon)\geq 1-\delta$

pac identification is used if EX( ) is available. Otherwise we use exact identification


%%%%%%%%%%%%%%%


Equivalence query

exhaustive search: enuerate indeices $i=1,\cdots,$ quering each $L_i$ until one gets an answer of $yes$, and halts at this point.

If hypothesis space is too large, then exhaustive search may use too many times of equivalence queries. 


$U$ binary strings of length $n$. $S=U$. $L_*$ is a unknown string want to learn.
Only equivalece, membership, subset, disjoint queries are available, then in the worst case, may need $2^{n}-1$ queries.
\\

Majority strategy (me want to learn some thing from an adversary). an input of the equivalent query can be not a hypothesis. 
Then log$N$ queries is sufficient. 

%%%%%%%
Lower bound, and techniques.

input must be a hypothsis.



hypothesis $L_1, \cdots, L_N$ such that $L_i\cap L_j=L_{\cap}$ for $i\not=j$. 

To learn $L_{\cap}$ 


\end{document}