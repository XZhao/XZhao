\documentclass[12pt]{article}
\usepackage{graphicx}
\usepackage{makeidx}
\usepackage{amsmath}
\usepackage{amsfonts}
\usepackage{color}
\usepackage[all]{xy}
\usepackage{CJK}
%\usepackage{ctex}

%%%%%%%%%%%%%%%%%%%%%%%%%%%%%%%%%%%%%%%%%%%%%%%%%%%%%%%%%%%%
\long\def\remove#1{}
\newcommand{\dom}{\mbox{dom}}

\newcommand{\vp}{\varphi}


\newcommand{\coNP}{\mbox{coNP}}

\newcommand{\NP}{\mbox{NP}}
\newcommand{\DP}{\mbox {D}^P}

\newcommand{\LeKl}{\mbox{$\sqcup$ }}
\newcommand{\card}[1]{\mbox{\#}(#1)}

\newcommand{\sss}{\mbox{\bf S}}
\newcommand{\jjj}{\mbox{\bf J}}

\newcommand{\QR}[1]{\mbox{$\ \mid\!\!\!\frac{#1}{\
    \stackrel{\mbox{\scriptsize\it Q-Res}}{\ }\ }\ $}}
\newcommand{\QUR}[1]{\mbox{$\ \mid\!\!\!\frac{#1}{\
    \stackrel{\mbox{\scriptsize\it Q-Pos-Unit-Res}}{\ }\ }\ $}}
%\baselineskip 0.2in


\newcommand{\pbox}{\hbox to 6pt{\leaders\hrule width 6pt height 6pt\hfill}}

\newtheorem{definition}{Definition}
\newtheorem{theorem}{Theorem}
\newtheorem{lemma}{Lemma}
\newtheorem{corollary}{Corollary}
\newtheorem{proposition}{Proposition}
\newenvironment{proof}{\parindent=0pt{\bf Proof: }}{
   \hspace*{\fill}\hbox to 6pt{\leaders\hrule width 6pt height 6pt\hfill}\par}


\pagestyle{plain}

\begin{document}

%\begin{CJK*}{GBK}{song}
%\CJKtilde



%%%%%%%

\title{
}

\author{
Xishun Zhao \footnote{Corresponding author. Tel: 0086-20-84114036,
Fax:0086-20-84110298.}
%\thanks{This research was partially supported by the NSFC project
%under grant number: 60970040 and a MOE project
%under grant number: 05JJD72040122. }\\
%Institute of Logic and Cognition,
%\\ Sun Yat-sen
%University\\ 510275 Guangzhou, (P.R. China)\\
%{Email: hsszxs@mail.sysu.edu.cn}
}



\maketitle

\begin{abstract}

\end{abstract}




Fix a formula $F$. Let 
$$h_1:=f_1\textsf{U}g_1,\cdots, h_k:=f_k\textsf{U}g_k$$ be all $\textsf{U}$-subformulas of $F$.\\

Pick new atoms $c_1, \cdots, c_k$\\

An outer-most occurrence of $f_i\textsf{U}g_i$ in $F$ is an occurrences not in another $U$-formulas. \\

$F^C$ be the formula obtained from $F$ by replacing each outer-most occurrence of $f_i\textsf{U}g_i$ by $c_i$. 


Now in $F^C$ there are no occurrences of $\textsf{U}$-formulas.\\ 

 
CL$(F^C)$ be as usual. Let $n$ be the $\textsf{X}$-depth of $F^C$ is $n$. Define CL$^m(F^C)$ be the formulas in $F^C$ with $\textsf{X}$-depth $\leq \max\{0, n-m\}$.  

\begin{definition}
$$s_0, s_1, \cdots$$
Define 

$$S,i\models g \ \ \mbox{ iff }\ \ S,i\models g[c_1/h_1,\cdots,c_k/h_k]$$



$sat_S(i,i):=\{g\in CL(F^C)\mid S,i\models g\}$

$sat_S(i,i+m):=\{g\in CL(F^C)\mid g\in CL^m(F^C), S,i\models g\}$

\end{definition}


\begin{lemma}
$S: s_0, s_1, \cdots, s_i, s_{i+1}, \cdots, s_j, \cdots,$

$S': s_0, \cdots, s_{i-1}, s_j, \cdots, $

Then 

(1) for every $m <i$ and every formula $g\in CL^m(F^C)$, 

$S,m\models g$ iff $S',m\models g$\\

(2) for every $m\geq i$ and every formula $g\in CL(F^C)$

$S, j+m\models g$ iff $S', i+m\models g$

\end{lemma}



\end{document}