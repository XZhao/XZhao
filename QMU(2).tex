\documentclass[12pt]{article}
\usepackage{graphicx}
\usepackage{makeidx}
\usepackage{amsmath}
\usepackage{amsfonts}
\usepackage{color}
\usepackage[all]{xy}
\usepackage{CJK}


%%%%%%%%%%%%%%%%%%%%%%%%%%%%%%%%%%%%%%%%%%%%%%%%%%%%%%%%%%%%
\long\def\remove#1{}
\newcommand{\dom}{\mbox{dom}}

\newcommand{\vp}{\varphi}


\newcommand{\coNP}{\mbox{coNP}}

\newcommand{\NP}{\mbox{NP}}
\newcommand{\DP}{\mbox {D}^P}

\newcommand{\LeKl}{\mbox{$\sqcup$ }}
\newcommand{\card}[1]{\mbox{\#}(#1)}

\newcommand{\sss}{\mbox{\bf S}}
\newcommand{\jjj}{\mbox{\bf J}}

\newcommand{\QR}[1]{\mbox{$\ \mid\!\!\!\frac{#1}{\
    \stackrel{\mbox{\scriptsize\it Q-Res}}{\ }\ }\ $}}
\newcommand{\QUR}[1]{\mbox{$\ \mid\!\!\!\frac{#1}{\
    \stackrel{\mbox{\scriptsize\it Q-Pos-Unit-Res}}{\ }\ }\ $}}
%\baselineskip 0.2in


\newcommand{\pbox}{\hbox to 6pt{\leaders\hrule width 6pt height 6pt\hfill}}

\newtheorem{definition}{Definition}
\newtheorem{theorem}{Theorem}
\newtheorem{lemma}{Lemma}
\newtheorem{corollary}{Corollary}
\newtheorem{proposition}{Proposition}
\newenvironment{proof}{\parindent=0pt{\bf Proof: }}{
   \hspace*{\fill}\hbox to 6pt{\leaders\hrule width 6pt height 6pt\hfill}\par}


\pagestyle{plain}

\begin{document}

\begin{CJK*}{GBK}{song}
\CJKtilde



%%%%%%%

\title{
}

\author{
Xishun Zhao \footnote{Corresponding author. Tel: 0086-20-84114036,
Fax:0086-20-84110298.}
%\thanks{This research was partially supported by the NSFC project
%under grant number: 60970040 and a MOE project
%under grant number: 05JJD72040122. }\\
%Institute of Logic and Cognition,
%\\ Sun Yat-sen
%University\\ 510275 Guangzhou, (P.R. China)\\
%{Email: hsszxs@mail.sysu.edu.cn}
}


\maketitle

\begin{abstract}

\end{abstract}
%%%%%%%%%%%%%%%%%%%

Given $f, g$, and $H$ such that $f+H\in$ MU(1), $g+H\in$ MU(1). Then 

$$\forall h\in H, \ (H-h)+\{f,g\} \text{ is satisfiable}$$

Proof. For $n=1$, it is clearly true. 

Then by induction and the disjoint splitting propery of MU(1) formulas \\



Suppose $\Phi:=Q\varphi$ with $\varphi=H+\{f,g\}$ with the above property.
Then (conjecture)
%
$\Phi$ is true iff both $Q(H+f)$ and $Q(H+g)$ are ture\\


Need the following property. 

$F\in $MU(1). $f, g \in F$, a path from $f$ to $g$ 

$f_1=f, \cdots, f_n=g$ and $L_1, \cdots, L_n$ such that 

$L_1\in f_1$, $\neg L_i, L_{i+1}$ are in $f_{i+1}$ for $i=1, \cdots n-1$, $\neg L_n \in f_n$. \\


Claim: Let $f,g\in F$, $\pi_1, \pi_2$ are two paths from $f$ to say to $g_1$ and $g_2$. Then the two path are compatable, that is, they do not contain complementary literals. 

Proof. For $n=1$ clearly. For $n>1$ by using disjoint splitting. 


{\bf So, in this case, QMU(2) is solvable in polynomial time.}


Suppose $F$ is lean $d(F)=2$. 

Case 1. $F-\{f\}\in$ MU(1), and for all $h\in F-\{f\}$, $F-\{h\}$ is satisfiable.

Case 2. $F-\{f\}\in$ MU(1), and there is $g\in F-f$, $F-g$ is in MU(1) (then this case)

Case3. $F'\subseteq F\in $MU(1) such that $F-F'$ has more than one clauses. 

Case 4. $F\in$ MU(2)

%%%%%%%%%%%%%%%%%%%%%%%%%%%%%%%%%%


\section{General Structure of LEAN(2)}

Suppose $F\in\text{MLEAN}(2)$ unsatisfiable, but $F\not\in \text{MU}(2)$. 

Then $F$ must contain a MU(1) subformula. So, let $G\subseteq F$ be a subformula in MU(1). Let $\theta=F-G$. 

Plaese note that $F$ is matching lean, the formula $\theta'$ obtained from $\theta$ by omitting variables occuring in $G$ must be mlean and has deficiency 1.  

{\bf Case 1.} For any clause $g\in G$  such that $(G-\{g\})+\theta$ is satisfiable. That is, $G$ is the only MU(1) subformula of $F$. 

We suppose it is not case 1. 

   
{\bf Case 2.} For some $g\in G$ such that $(G-\{g\})+\theta$ is unsatisfiable.
That is there is $G'\subseteq F$ such that $G'\not=G$ and $G'\in \text{MU}(1)$. 

Because $G'=G'\cap G+G'\cap \theta$ and $G'$ has deficiency 1, it must be that $\theta\subseteq G'$. 

Let $H=G\cap G'$, $\eta =G'-H$. That is, $F$ can be written as follows

$$F=\theta+H+\eta\ \text{ with }\theta+H\in \text{MU}(1) \text{ and } \eta+H \in \text{MU}(1)$$ 

Let $\theta^-$ (resp. $\eta^-$) be the formula obtained from $\theta$ (resp. $\eta$) by omitting occurrences of variables of $H$. Then both $\theta^-$ and $\eta^-$ are in MU(1) and have distinct variables. 

The following is an example of such formulas \\


Let $\theta^+$ is the clause obtained from $\theta$ by iteratively applying sDP reduction on variables in $\theta^-$. Likewise for $\eta^+$. Then we have 

$$\{\theta^+\}+H,\ \{\eta^+\}+H \text{ and } \{\theta^+\vee \eta^+\}+H \text{ are all in MU(1)}$$ 

%%%%%%%%

We consider the first case. That $F$ contains only MU(1) subformula.



\end{CJK*}
\end{document}