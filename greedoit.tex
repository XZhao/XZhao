\documentclass[12pt]{article}
\usepackage{graphicx}
\usepackage{makeidx}
\usepackage{amsmath}
\usepackage{amsfonts}

%%%%%%%%%%%%%%%%%%%%%%%%%%%%%%%%%%%%%%%%%%%%%%%%%%%%%%%%%%%%
\long\def\remove#1{}
\newcommand{\dom}{\mbox{dom}}

\newcommand{\vp}{\varphi}


\newcommand{\coNP}{\mbox{coNP}}

\newcommand{\NP}{\mbox{NP}}
\newcommand{\DP}{\mbox {D}^P}

\newcommand{\LeKl}{\mbox{$\sqcup$ }}
\newcommand{\card}[1]{\mbox{\#}(#1)}

\newcommand{\sss}{\mbox{\bf S}}
\newcommand{\jjj}{\mbox{\bf J}}

\newcommand{\QR}[1]{\mbox{$\ \mid\!\!\!\frac{#1}{\
    \stackrel{\mbox{\scriptsize\it Q-Res}}{\ }\ }\ $}}
\newcommand{\QUR}[1]{\mbox{$\ \mid\!\!\!\frac{#1}{\
    \stackrel{\mbox{\scriptsize\it Q-Pos-Unit-Res}}{\ }\ }\ $}}
%\baselineskip 0.2in

\newcommand{\AK}{\mbox{${\cal{A}}_K$}}

\newcommand{\pbox}{\hbox to 6pt{\leaders\hrule width 6pt height 6pt\hfill}}

\newtheorem{definition}{Definition}[section]
\newtheorem{theorem}{Theorem}[section]
\newtheorem{lemma}{Lemma}[section]
\newtheorem{corollary}{Corollary}[section]
\newtheorem{proposition}{Proposition}[section]
\newtheorem{example}{Example}[section]
\newtheorem{remark}{Remark}[section]
\newenvironment{proof}{\parindent=0pt{\bf Proof: }}{
   \hspace*{\fill}\hbox to 6pt{\leaders\hrule width 6pt height 6pt\hfill}\par}


\pagestyle{plain}

\begin{document}

Hyper graph $G=(V,E)$ tr$(G)$ the set of minimal transversals of $G$.

If $G=(V, G)$ then $E(G)=E$.

\section{Greedoid}

in Section 2 of the full report\\

greedoid is a hypergraph $(V,G)$ with the following properties:

\begin{enumerate}
\item $E$ is not empty.

\item $E$ is accessible (to the empty edge), for each $H\in E$ s.t. $H\not=\emptyset$, there is $v\in V$ such that $(H-\{v\})\in E$ 

\item $E$ is augmentable (shorter edges can be made longer). for two edges $A,B$ with $|A|<|B|$, there is $v\in B-A$ such that $A\cup\{v\}\in E$.
\end{enumerate}


A basis is maximal super edge  in $E$ wrt set-inclusion. Clearly, all basis have the same length due the aumenttability. A basis is edges with longest length.\\


A matiod is a heredatary greedoid, if $H$ is an edge then its any subset also is an edge. \\


interval greedoid $G=(V,E)$: For edges $A,B,C$ and vertex $x$, $$A\subseteq B\subseteq C\mbox{ and }A\cup\{x\}, C\cup\{x\}\in E \Longrightarrow B\cup \{x\}\in E$$ 


antigreedoid $G=(V,E)$: For edges $A,B$ and vertex $x$, $$A\subseteq B\mbox{ and }A\cup\{x\}\in E \Longrightarrow B\cup \{x\}\in E$$

antigreedoid is an interval one\\

Gaussian greedoid $G=(V,E)$: For $X, Y$ with $|X|=|Y|+1$, 
$$X, Y\in E \Longrightarrow \exists x\in X-Y, \mbox{ s.t. }X-\{x\}, Y\cup\{x\}\in E$$


\section{singular Tuples}

in Section 5.2 of the full report.\\


singular tuple $(v_1,\cdots, v_m)$ for $F$.

$v_1$ is singular in $F$ and, $v_{i+1}$ is singular in DP$_{v_1,\cdots,v_i}(F)$.

segment of a singular tuple is also singular.

1-singular (once positive and once negative)  $m$-singular (non-1-singular)

$(v_1,\cdots, v_n)$ is 1-singular if every $v_i$ is 1-singular

is non-1-singular if every $v_i$ is non-1-singular.\\


Totally singular tuple $(v_1,\cdots, v_n)$ if for every permutation $\pi$, $(v_{\pi(1)},\cdots, v_{\pi(n)})$ is also a singular tuple.

about totally singular see Subsection 5.4

Conjecture: If $(v_1,\cdots, v_n)$ is non-1-singular then it is totally singular.\\

More question see 5.27, 5.37.

\section{Hypergraph of singular sets}


%%%section 5.5


ssh$(F)$ the hypergraph of singular sets of $F$.


vertices are variables, 

edges are $\{v_1,\cdots,v_n\}$ such that $(v_1,\cdots, v_n)$ is a singular tuple. that is, an edge can be a singular tuple by rearrange the order. 
\\

mss$(F)$ the hypergraph of maximal singular sets. 
consists of longest edges of ssh$(F)$. i.e. $H$ with $|H|=\mbox{si}(F)$\\


singular variable hypergraph svh$(F)$:

edges: $\{\mbox{var}(x)\mid x\in C \mbox{ and }x \mbox{ is singular in }F\}$ for all $C\in F$ with this set non-empty.





\end{document}  