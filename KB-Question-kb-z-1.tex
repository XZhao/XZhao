\documentclass[12pt]{article}
\usepackage{graphicx}
\usepackage{makeidx}
\usepackage{amsmath}
\usepackage{amsfonts}
\usepackage{color}
\usepackage[all]{xy}


%%%%%%%%%%%%%%%%%%%%%%%%%%%%%%%%%%%%%%%%%%%%%%%%%%%%%%%%%%%%
\long\def\remove#1{}
\newcommand{\dom}{\mbox{dom}}

\newcommand{\vp}{\varphi}


\newcommand{\coNP}{\mbox{coNP}}

\newcommand{\NP}{\mbox{NP}}
\newcommand{\DP}{\mbox {D}^P}

\newcommand{\LeKl}{\mbox{$\sqcup$ }}
\newcommand{\card}[1]{\mbox{\#}(#1)}

\newcommand{\sss}{\mbox{\bf S}}
\newcommand{\jjj}{\mbox{\bf J}}

\newcommand{\QR}[1]{\mbox{$\ \mid\!\!\!\frac{#1}{\
    \stackrel{\mbox{\scriptsize\it Q-Res}}{\ }\ }\ $}}
\newcommand{\QUR}[1]{\mbox{$\ \mid\!\!\!\frac{#1}{\
    \stackrel{\mbox{\scriptsize\it Q-Pos-Unit-Res}}{\ }\ }\ $}}
%\baselineskip 0.2in

\newcommand{\AK}{\mbox{${\cal{A}}_K$}}
\newcommand{\AM}{\mbox{$^\exists\!\!\models$}}
%\newcommand{\AMB}{\mbox{$^A$\hspace{-0.5mm}$\models^B$}}
\newcommand{\AMB}{\mbox{$^\exists\!\!\models^\exists$}}

%\newcommand{\AM}{\mbox{$^A$\hspace{-0.5mm}$\models$}}

\newcommand{\AEQB}{\mbox{$^\exists\!\!\approx^\exists$}}
\newcommand{\AEQ}{\mbox{$^\exists\!\!\approx$}}

\newcommand{\pbox}{\hbox to 6pt{\leaders\hrule width 6pt height 6pt\hfill}}

\newtheorem{definition}{Definition}[section]
\newtheorem{theorem}{Theorem}[section]
\newtheorem{lemma}{Lemma}[section]
\newtheorem{corollary}{Corollary}[section]
\newtheorem{proposition}{Proposition}[section]
\newtheorem{example}{Example}[section]
\newtheorem{remark}{Remark}[section]
\newenvironment{proof}{\parindent=0pt{\bf Proof: }}{
   \hspace*{\fill}\hbox to 6pt{\leaders\hrule width 6pt height 6pt\hfill}\par}


\pagestyle{plain}

\begin{document}

\section{Configuration}
recall restricted equivalence
\begin{definition}
Let $\alpha$ and $\beta$ be propositional formulas and $V$ a set of variables.
If $\{\pi : \alpha \models \pi, var(\pi) \subseteq V\} \subseteq \{ \sigma : \beta \models \sigma, var(\sigma) \subseteq V\}$ then we write $\alpha \models_{V} \beta$. If $\alpha \models_V \beta$ and $\beta \models_V \alpha$ then we say that $\alpha$ is $V$-equivalent to $\beta$, in symbols
$\alpha \approx_V \beta$.
\end{definition}


relationship to existential variables
\begin{lemma}
Let $\alpha$ be a propositional formula over the variables $Y \cup A$ and $\beta$ be a formula over the
$X \cup B$, where $(Y \cup X) \cap (A \cup B)$ is empty. Then we have: $(\exists Y \alpha) \models (\exists X \beta)$
($\exists Y \alpha) \approx (\exists X \beta)$ respectively) if and only if $\alpha \models_{A \cup B} \beta$
($ \alpha \approx_{A\cup B} \beta$ respectively).

\end{lemma}
\begin{definition}(Configuration Problem for $\models$, \AM, and \AMB)\\
 Let $R$ and $T$ be classes of propositional formulas. Then we define\\
{\em Problem:  $R \Longrightarrow_{conf} T$-Implication for $\models$. }\\\
{\em Instance}: Set of components $C=\{\alpha_1,\cdots,\alpha_n\} \subseteq R$, target formula $\beta \in T$.\\
{\em Query}: $ \exists K \subseteq C: K \in$ SAT and $K \models \beta$?\\

\noindent{\em Problem:  $R \Longrightarrow_{conf} T$-Implication for \AM.}\\
{\em Instance}: Set of components $C=\{\alpha_1,\cdots,\alpha_n\} \subseteq R$ over the variables $X \cup A$, target formula $\beta \in T$.\\
{\em Query}: $ \exists K \subseteq C: K \in$ SAT and $(\exists X K) \models \beta$?\\

\noindent{\em Problem:  $R \Longrightarrow_{conf} T$-Implication for \AMB.}\\
{\em Instance}: Set of components $C=\{\alpha_1,\cdots,\alpha_n\} \subseteq R$ over the variables $X \cup A$ and the target formula $\beta \in T$ over the variables $Y \cup B$.\\
{\em Query}: $ \exists K \subseteq C: K \in$ SAT and $(\exists X K) \models (\exists Y \beta)$?\\
\end{definition}








\section{Implication}
\begin{definition}
Let $\alpha$ be a formula over the variables $X \cup A$ and let $\beta$ be a formula over the variables $Y \cup B$.\\
We write $\alpha $ \AMB $\beta$ ($\alpha $\AM $\beta$ resp.) if and only if
$(\exists X \alpha) \models (\exists Y \beta)$ ($\exists X \alpha \models \beta$ resp.).
\end{definition}

\color{red}
\noindent Remark: When we use $X,Y$ and $A, B$, we have to specify them. Then a problem occurs that are they fixed for all $\alpha, \beta$ or they depends on $\alpha, \beta$ (I think they can change). So I suggest the following notations.\\

\noindent {\bf Definition 1.1}\
Given two formulas $\alpha,\beta$, let $X=\mbox{var}(\alpha)-\mbox{var}(\beta)$ and $Y=\mbox{var}(\beta)-\mbox{var}(\alpha)$. We define
\begin{enumerate}
\item $\alpha\, ^\exists\!\!\models^\exists \beta$ if and only if $\exists X\alpha\models \exists Y\beta$

\item $\alpha\, ^\exists\!\!\models \beta$ if and only if $\exists X \alpha\models \beta$.
\end{enumerate}
\color{blue}
Definition 1.1 leads to problems: Suppose we have a formula $\alpha$ with variables $a_1, a_2, a_3, x_1, x_2$ and a formula $\beta$ with variables $a_1, a_2$. Now we want to know whether or not $\exists x_1 \exists x_2 \alpha \models \beta$. $\ \ y_1,y_2$ are the helping variables.\\
With your definition 1.1 we can only deal with $X= var(\alpha) - var(\beta) = \{x_1, x_2, a_3\}$ and then
with
$\exists x_1 \exists  x_2 \exists a_3 \alpha \models \beta$. That is another question, because we have $a_3$ additionally in the prefix.\\
Your definition is more restrictive and from my point of view would not cover the intention of configuration systems.

My intention for my definition was to adapt the definition of restricted equivalence. For example, $\alpha \approx_{V}
\beta$ says that for a fixed set $V$ they have the same set of consequences. $\alpha$ \AMB $\beta$ means that for sets of
variables $A$ and $B$ we have $\exists (\mbox{var}(\alpha) - A) \alpha \models \exists (\mbox{var}(\beta) - B) \beta$.
\color{black}


\begin{definition}
 Let $R$ and $T$ be classes of formulas. Then we define\\
{\em Problem:  $R \Longrightarrow_{conf} T$-Implication for $\models$, \AM, \AMB}\\
{\em Instance}: A set of components $C=\{\alpha_1,\cdots,\alpha_n\} \subseteq R$ over the variables $X \cup A$ and the target formula $\beta \in T$ over the variables $Y \cup B$. In order to avoid renaming we assume that $Y \cap A= X \cap B=$ is empty.\\
{\em Query}:
 $ \exists K \subseteq C: K \in$ SAT and $K \models \beta$ $((\exists X K)  \models \beta, \
(\exists X K) \models (\exists Y \beta)$ respectively)?\\
\end{definition}

\color{red}
\noindent {\bf Definition 1.2}\ Let $R$ and $T$ be two classes of formulas. Then we define the configuration implication problem
$(R, T, \models)$ (resp. $(R, T, ^\exists\!\models)$, $(R, T,^\exists\!\models^\exists$)) as follows:

\begin{description}
%\item {Problem:} $(R, T, \models)$ (resp. $(R, T, ^\exists\!\models)$, $(R, T,^\exists\!\models^\exists$))
\item {Instance:} A set of components $C=\{\alpha_1,\cdots, \alpha_n\}\subseteq R$ and a target formula $\beta\in T$.
\item {Query:} Does there exist $K\subseteq C$ such that $K\in \mbox{SAT}$ and $K\models \beta$ (resp. $K^\exists\!\!\models \beta$, $K
^\exists\!\!\models^\exists\beta$)?
\end{description}
\color{black}


\begin{theorem}
The table contains the computational complexities of\\ $R \Longrightarrow_{conf} T$-Implication problems.\\
\begin{tabular}{|l|l|l|l|l|}
\hline
 & {\em Classes} &  $\models$ & \AM & \AMB \\ \hline
1. & DHORN $\Rightarrow_{conf}$ DHORN          & PTIME & PTIME  & coNP-c         \\ \hline
2. & DHORN $\leq k$ $\Rightarrow_{conf}$ DHORN & NP-c  & NP-c   & ---         \\ \hline
3. & 2-Term $\Rightarrow_{conf}$ Term          & NP-c  & NP-c   & NP-c           \\ \hline
4. & HORN $\models$ HORN                       & NP-c  & NP-c   & open $\Sigma^p_2$-c \\ \hline
5. & CLAUSE $\Rightarrow_{conf}$ LITERAL       & $\Sigma_2^P$-c & $\Sigma_2^P$-c & $\Sigma_2^P$-c \\ \hline
6. & CNF $\Rightarrow_{conf}$ CLAUSE, LITERAL  & $\Sigma_2^P$-c & $\Sigma_2^P$-c & $\Sigma_2^P$-c \\ \hline
7. & CNF $\Rightarrow_{conf}$ CNF              & $\Sigma_2^P$-c & $\Sigma_2^P$-c & open $\Sigma_3^P$-c \\ \hline
\end{tabular}
\end{theorem}

\color{red}
\noindent If you agree the notation. The above table would be.\\

\noindent\begin{tabular}{|l|l|l|l|l|l|}
\hline
 & {Class $R$ } & Class $T$ &  $(R,T,\models)$ & $(R, T, ^\exists\!\models)$ & $(R, T,^\exists\!\models^\exists$) \\ \hline
1. & DHORN & DHORN          & PTIME & PTIME  & coNP-c         \\ \hline
2. & DHORN $\leq k$ & DHORN & NP-c  & NP-c   & ---         \\ \hline
3. & 2-Term & Term          & NP-c  & NP-c   & NP-c           \\ \hline
4. & HORN & HORN                       & NP-c  & NP-c   & open $\Sigma^p_2$-c \\ \hline
5. & CLAUSE & LITERAL       & $\Sigma_2^P$-c & $\Sigma_2^P$-c & $\Sigma_2^P$-c \\ \hline
6. & CNF & CLAUSE, LITERAL  & $\Sigma_2^P$-c & $\Sigma_2^P$-c & $\Sigma_2^P$-c \\ \hline
7. & CNF & CNF              & $\Sigma_2^P$-c & $\Sigma_2^P$-c & open $\Sigma_3^P$-c \\ \hline
\end{tabular}





\color{black}




\vspace*{5mm}

\begin{proof}
Ad 1: (DHORN, $\models$ and \AEQ): Since every definite Horn formula is satisfiable, we only we have to test whether for the complete set of components $C$ we have $C \models \beta$ or $\exists X C \models \beta$. That can be performed in polynomial time.\\

(DHORN,\AMB): well-known trick\\

Ad 2: (DHORN $K$, $\models$ and \AM): Next we show the NP-completeness of the problem whether for a set of components $C$, a target formula $\beta$ in DHORN, and a fixed $k$ there exists some $K \subseteq C$ with $K \models \beta$ and at most $k$ components.\\
Since $K \models \beta$ is in polynomial time decidable, the problem is in NP (guess a subset $K \subseteq C$).
The NP-hardness can be shown by a reduction to the Minimal Input Set Problem (MI) for definite Horn formulas.
Let $\alpha$ be a definite Horn formula without unit clauses, and let $H$ be the set of variables occurring only
negatively in $\alpha$, and $y$ be a variable which occurs positively in $\alpha$. Then we have to decide whether
there exists a subset $I \subseteq H$ with at most $k$ variables, such that $I \wedge \alpha \models y$. This problem is known to be NP-complete \cite{mi}.

We can associate to $\alpha$ and $y$ a configuration problem as follows: We define $C = H \cup \{\alpha\}$, the target
formula is $y$. If there is a solution for the MI-problem then there is subset $K \subseteq C$ with at most $k+1$ elements
and $K \models y$. For the other direction, suppose there exists $K \subseteq C, |K | \leq k+1$ and $K \models y$.
\end{proof}

\begin{proof}
AD 3:  (Term,$\models$, \AM, \AMB): The components are terms with at most 2 literals and target formula is a term or an existentially quantified term, which can be reduced to a term. Obviously, the problems are in NP (guess $K$, the test
is simple). Now we will show the NP-hardness.

Let $\alpha = \alpha_1 \wedge \ldots \wedge \alpha_m$ be a 3-CNF formula over the variables $x_1, \ldots, x_n$. We associate
to $\alpha$ a set of 2-term components $C$, a target formula $\beta= q_1 \wedge \ldots \wedge q_m$, such that $\alpha$ is satisfiable if and only if
there exists $K \subseteq C$ for which $K$ is satisfiable and $K \approx\beta$.

Let $q_1, \ldots, q_m$ be new variables for $\alpha$.\\
We define $C=\{(q_i \wedge L) : 1 \leq i \leq m, L \in \alpha_i\}$. The variable $q_i$ indicates that $L$ occurs in the clause $\alpha_i$.\\
Suppose, $\alpha$ is satisfiable. Then there is a truth assignment $v$ with $v(\alpha)=1$ and therefore for every clause $\alpha_i$ a literal $L$ with $v(L)=1$.\\
We define $K = \{(q_i \wedge L) : 1 \leq i \leq m, v(L)=1, L \in \alpha_i\}$. Then $K$ is satisfiable and since
for every clause $\alpha_i$ there exists some $L \in \alpha_i$ with $v(L)=1$ we obtain $K \models q_1 \wedge \ldots \wedge q_m$.\\
For the other direction we assume that there is some $K \subseteq C$ with $K \in$ SAT and $K \models \beta$.
Then for every variable $q_i$ there is a term $q_i \wedge L$ in $K$. Since $K$ is satisfiable, the conjunction of literals $L$ of the components (2-terms) are satisfiable. This leads to a satisfying truth assignment for $\alpha$.
\end{proof}

\begin{proof}
(HORN, $\models$, \AM):\\
Since every term is a Horn formula, the NP-hardness follows from part (3).
 Whether or not $\exists X K) \models \beta$ can be decided in polynomial time. Therefore, we only have to guess a satisfiable subset $K$ of $C$. That shows that the problems belong to NP.\\

(HORN, \AMB):  That the problems are in $\Sigma^p_2$ is obvious. We can guess some satisfiable $K \subseteq C$ and
then to decide the coNP-problem $(\exists X K \models (\exists Y \beta)$. For the complementary problem guess a
truth assignment $v$ for the free variable and check whether $v(\exists X K)=1$ and $v(\exists Y \beta)=0$.
That can be performed in polynomial time , because we are dealing with quantified Horn formulas. Altogether,
we have $\Sigma^p_2$-procedure.\\

Now we have to how the $\Sigma^p_2$-hardness.

\color{red}

Consider an arbitrary $\Phi:=\exists U\forall W (\theta\vee\sigma)$ such that $\theta, \sigma\in$ 3-DNF, and that $\theta$ is negatively monotone
while $\sigma$ is positively monotone.
%
Let $$\begin{array}{l}\theta=\bigvee\{(\neg L_{i,1}\wedge \neg L_{i,2}\wedge \neg L_{i,3})\mid i=1,\cdots,n\},\\ \sigma=\bigvee\{(K_{j,1}\wedge K_{j,2}\wedge K_{j,3})\mid j=1,\cdots,m\},\end{array}$$
where $L_{i,p}$ and $K_{j,q}$ are variables from $U\cup W$.\\

We pick new variables $a_1,\cdots, a_n$, and $b_1,\cdots, b_m$, and $c$. Define

$$\begin{array}{cll}C_1&:=&\{a_1\wedge\cdots\wedge a_n\wedge b_j\rightarrow c\mid 1\leq j\leq m\}\cup\\
&&\{L_{i,p}\rightarrow a_i\mid 1\leq i\leq n, 1\leq p\leq 3\}\cup\\
&&\{K_{j,1}\wedge K_{j,2}\wedge K_{j,3}\rightarrow b_j\mid 1\leq j\leq m\}\\
C_2&:=&U\cup\{\neg u\mid u\in U\}\\
C&:=&C_1\cup C_2\\ \\
\beta&:=&\{a_1\wedge\cdots\wedge a_n\rightarrow c\}\cup\\
&&\{L_{i,p}\rightarrow a_i \mid 1\leq i\leq n, 1\leq p\leq 3\}
\end{array}$$

Let
$$X:=\{a_1,\cdots, a_n, b_1,\cdots, b_m\}, \ \ Y:=\{a_1,\cdots, a_m\}$$


Next we shall show: $\Phi$ is true iff $\exists K\subseteq C$ satisfiable $\exists X K\models \exists Y\beta$\\

Suppose $\Phi$ is true, then there is a truth assignment $t$ on $U$ such that $\forall W(\theta\vee \sigma)[t]$ is true.
Define
$$K:=C_1\cup\{u\mid t(u)=1, u\in U\}\cup\{\neg u\mid t(u)=0, u\in U\}$$
Obviously, $K$ is satisfiable. We show that $\exists X K\models\exists X\beta$. Consider any truth assignment $s$ satisfying $\exists X K$. Then clearly $s$ and $t$ agree on $U$.
There must be an extension $s'$ of $s$ such that $s'$ satisfies $K$.  Since  $\forall W(\theta\vee \sigma)[t]$ is true, we have two cases:\\

{\bf Case 1.} $\neg L_{i,1}\wedge\neg L_{i,2}\wedge \neg L_{i,3}$ is true under $s'$ for some $i$. Then we modify $s'$ to obtain $s''$ as follows:
%
$s''(a_i)=0, s''(a_h)=s'(a_h)$ for $h\not=i$, and $s''(c)=s'(c)$. It is easy to see that $s''$ satisfies $\beta$. Therefore $s$ satisfies $\exists Y\beta$ since $s, s', s''$ agree on variables in $U\cup W\cup\{c\}$.\\

{\bf Case 2.} Not Case 1. There must be some $j$ such that $K_{j,1}\wedge K_{j,2}\wedge K_{j,3}$ is true under $s'$. Then $s'(b_j)=1$. Because Case 1 does not happen, all $a_1,\cdots,a_n$ are true under $s'$. Hence, $s'(c)=1$. Consequently $s'$ satisfies $\beta$. Thus, $s$ satisfies $\exists Y\beta$.\\


Altogether, we have $\exists X K\models \exists Y \beta$. \\


For the inverse direction we suppose there is $K\subseteq C$ such that $K$ is satisfiable and $\exists X K\models \exists Y \beta$.
We want to show that $\exists U\forall Y (\theta\vee \sigma)$ is true.\\

Please note that for any satisfiable $K'\subseteq C_2$, $C_1\cup K'$ is satisfiable. Thus we may assume that $C_1\subseteq K$.\\


{\bf Case 1.} Either $u\in K$ or $\neg u\in K$ for any variable $u\in U$. Let $t$ be the truth assignment defined by $t(u)=1$ iff $u\in K$. We shall show $\forall W(\theta\vee \sigma)[t]$ is true. Suppose otherwise, then $t$ has an extension $t'$ which makes $\neg\theta \wedge \neg \sigma$ true.
That is, for all $i=1,\cdots, n$ there is some $p=1,2,3$ such that $L_{i,p}$ is true under $t'$, and for all $j=1,\cdots,m$ there is some $q=1,2,3$ such that $\neg K_{j,q}$ is true under $t'$. Now we may extend $t'$ to $t''$ as follows: $t''(a_1)=\cdots=t''(a_n)=1$, $t''(b_1)=\cdots=t''(b_m)=0$, $t''(c)=0$. Clearly, $t''$ makes $K$ true. Let $s$ be the assignment obtained from $t$ by setting $c$ to be false. Then $s$ satisfies $\exists X K$ since $s$ and $t''$ agrees on $U\cup W\cup\{c\}$. However, $s$ can not be extended to satisfy $\beta$ because any extension of $s$ must make $a_1,\cdots, a_n$ and $c$ be true if it satisfies $\beta$. That is, $s$ does not satisfy $\exists Y\beta$ contradicts the assumption $\exists X K\models\exists Y\beta$.
Therefore, $\forall W(\theta\vee \sigma)[t]$ is true, so is $\exists U\forall W(\theta\vee \sigma)$.\\

{\bf Case 2.} Not Case 1. Let $U_1=:\{u\mid u\in K \mbox{ or }\neg u\in K\}$. Let $W_1:=W\cup (U-U_1)$. Now we consider $\exists U_1\forall W_1(\theta\vee\sigma)$. Then by the proof of Case 1, we obtain that $\exists U_1\forall W_1(\theta\vee\sigma)$ is true. Since $U_1\subseteq U, W\subseteq W_1$ it follows that $\exists U\forall W(\theta\vee\sigma)$ is also true.

\color{black}







\end{proof}
\vspace*{3mm}
\begin{proof}
AD 5 and 6: (CNF, CLAUSE $\Longrightarrow_{conf}$ CLAUSE, Literal; \AMB)\\
At first we show that the problems are in $\Sigma^p_2$. It suffices to prove this upper bound for \AMB
and the case that the components are CNF-formulas and the target formula is a clause
$\beta =(L_1 \vee \ldots \vee L_m)$. Since $\exists Y \beta$ can be reduced to true, we can distinguish two cases:

Case 1: A literal $L$ over some variable $y \in Y$ occurs in $\beta$. Then we have $\exists Y \beta = 1$. That means our configuration problem is the question whether
$\exists K \subseteq C: (\exists X K) \models 1$, which is always true.

\color{red} REMARK:
Not necessary. Generally we can assume every component in $C$ is satisfiable.
\begin{enumerate}
\item Suppose $C$ consists of only clauses. If there is one component containing some variables in $X$, then the existence of $K$ is trivial.  Otherwise we have to check whether $C$ contains a tautological clause.
\item Suppose generally $C$ consists of CNF formulas. Please note that even a component $\alpha$ is sat, $\exists X\alpha$ is not necessarily always true (e.g. $\exists x (x\vee y)\wedge (\neg x\vee y)$ is sat but not always true). Thus in this case we have to check whether there is a component $\alpha$ in $C$ such that $\exists X\alpha$ is always true. Please note that $\exists X\alpha$ is always true iff $\alpha^{X}$ is satisfiable, here $\alpha^{X}$ is obtained from $\alpha$ by removing all literals whose variable are not in $X$. Thus, the existence of desired $K$ is in NP.
\end{enumerate}
\color{black}
\ \ \\

Case 2: $\beta$ contains no $Y$-variables, then we obtain $\exists Y \beta = \beta$ and we proceed with $\beta$.\\
$\exists K \subseteq C: K$ is satisfiable and $(\exists X K) \models \beta$ iff
$\exists K \subseteq C: K \in %\overline
{\mbox{SAT}}$ and $\forall X (\neg K \vee \beta)$ is a tautology.
That leads to $\Sigma^p_2$-procedure: Guess a satisfiable $K$ and check the tautology.\\

Now it remains to prove the $\Sigma^p_2$-hardness for components given  as clauses and a target formula given as a literal.
At first we show the poly-time equivalence of the configuration problem where the components are CNF-formulas and
the problem where the components are clauses.
Let $\beta=z$ be the target formula.\\
Let $C=\{\alpha_1, \ldots, \alpha_n\}$ be a set of components, where $\alpha_i$ are CNF-formulas.
For each formula $\alpha_i= \alpha_{i,1} \wedge \ldots \wedge \alpha_{i,t_i}$  we introduce new variables $q^i_0, \ldots, q^i_{t_i}$ and associate to $\alpha_i$ the components $q^i_0, (\neg q^i_0 \vee \alpha_{i,1} \vee q^i_1), \ldots
(\neg q^i_{t_i-1} \vee \alpha_{i,t_i} \vee q^i_{t_i}), \neg q^i_{t_i}$.

Let $C_{cl}$ be the set of these clauses. Then there exists $K \subseteq C$: $K \in$ SAT and $K \models z$ if and only if
there exists $K^* \subseteq C^*$: $K^* \in$ SAT and $K^* \models z$.\\

Now it remains to show the $\Sigma_2^P$-hardness:\\
Let $\Phi:= \exists x_1, \ldots, x_n \forall y_1, \ldots, y_m \varphi$ be a closed formula, where
$\varphi$ is a propositional DNF-formula. For those formulas the satisfiability problem is $\Sigma^p_2$-complete.
For a new variable $z$ we define the set of components\\
$C= \{ \varphi \rightarrow z, x_1, \neg x_1, \ldots, x_n, \neg x_n\}$ and the target formula $\beta=z$. Please note, that $\varphi \rightarrow z$ can easily transformed into an equivalent CNF-formula of length linear in the length of
$\varphi$.

Then it holds:\\
$\Phi$ is true if and only if $\exists K \subseteq C: K \in$ SAT and $K \models z$.\\
From left to right:\\
Suppose $\Phi$ is true. Then there is a partial truth assignment $v(x_1)= \epsilon_1, \ldots v(x_n)= \epsilon_n$, such
that $\forall y_1 \ldots \forall y_m \varphi([x_1/\epsilon_1, \ldots, x_n/\epsilon_n])$ is true. The subset of components
$K= \{ \varphi \rightarrow z, x_1^{\epsilon_1}, \ldots, x_n^{\epsilon_n}\}$ is satisfiable. Assume that $K\not \models z$.
Then there is a truth assignment $v$ with $v(z)=0$, but $v(K)=1$. Then we obtain $v(\varphi([x_1/\epsilon_1, \ldots, x_n/ \epsilon_n]) \rightarrow z)=1$ and therefore $v(\varphi([x_1/\epsilon_1, \ldots, x_n/ \epsilon_n])=0$ as a contradiction.\\

From right to left: Suppose, $\exists K \subseteq C: K \in$ SAT and $K \models z$. Then $\varphi \rightarrow z$ is in $K$,
because $z$ occurs only in this component.
\end{proof}

\vspace*{3mm}
\begin{proof}
Ad 7: (CNF $\longrightarrow_{conf}$ CNF, $\models$, \AM):)
Because of part (6) it suffices to show that the problems belong to $\Sigma_2^p$.
(see part 5,6)\\
the hardness follows from (5,6)\\

(CNF $\Longrightarrow_{conf}$, \AMB):\\
Membership $\Sigma_3^p$:

\color{red} The problem of determining
$\exists K\models \exists Y\beta$ is generally in $\Pi^P_2$. Thus the existence of desired $K$ is in $\Sigma_3^P$.
\color{black}


Next we show the $\Sigma^p_3$-hardness.

\color{red}
the proof idea should be the same as the configuration problem for \AEQB
\color{black}

\end{proof}




\section{Equivalence}




--------------------------------------------------------------------------------------------------------------
\begin{theorem}
Equivalence\\

\begin{tabular}{|l|l|l|l|l|}
\hline
 & {\em Classes} &  $\approx$ & \AEQ & \AEQB \\ \hline
1. & DHORN $\Rightarrow_{conf}$ DHORN & PTIME & open & coNP-h\\ \hline
2. & 2-Term $\Rightarrow_{conf}$ Term & PTIME & NP-c & NP-c\\ \hline
3. & HORN  $\Rightarrow_{conf}$ HORN & PTIME & in $\Sigma^p_2$ &  $\Sigma^p_2$-c \\ \hline
4. & CL $\Rightarrow_{conf}$ LITERAL & coNP-c & $\Sigma^p_2$-h & $\Sigma^p_2$-h\\ \hline
5. & CNF $\Rightarrow_{conf}$ CNF & $P^{NP[\text{log}\, n]}$, $D^P$-h  & $\Sigma^p_3$-c& $\Sigma^p_3$-c \\ \hline

\end{tabular}
\end{theorem}
\vspace*{5mm}
\begin{proof}
AD 1.1 DHORN $\Rightarrow_{conf}$ DHORN for $\approx$):\\
Since every definite Horn is satisfiable, every configuration is satisfiable.\\
Let $F(\beta) =\{\alpha_i: \beta \models \alpha_i, 1 \leq i \leq n\}$ the set of derivable components.
If $F(\beta)$ is empty, then there is no configuration. Otherwise, we
have: $\exists K \subseteq C: K \approx \beta$ iff $F(\beta) \approx \beta$. Thus, it remains to proof whether $F(\beta) \models \beta$ holds, which can be performed in polynomial time.

1.2 DHORN $\Rightarrow_{conf}$ DHORN for\AEQ and \AEQB):\\
Since for two definite Horn formulas $\exists X \alpha$ and $\exists Y \beta$ and with free variables $V$ the problem whether or
not $\exists X \alpha \approx \exists Y \beta$ is equivalent to the restricted equivalence problem $\alpha \approx_V \beta$ and this problem is known to be coNP-complete \cite{kb}, we see that the configuration problem for \AEQB is at least coNP-hard.
If helping variables only occur in the configuration $K$, that is for \AEQ , then the computational complexity remains open.
It is not known whether for definite Horn formulas $\alpha$ and $ \beta$, $\exists X \alpha \approx \beta$ can be decided in polynomial time or whether the problem is coNP-complete.

%\color{red}
%The case for \AM. For $\alpha$, let $\alpha^X$ be obtained from $\alpha$ by deleting literals not from $X$. 
%$\alpha^{-X}$ obtained from $\alpha$ by deleting literals from $X$. Please note that Let $C^{-X}:=\{\alpha^{-X}\mid \alpha\in C\}$.
%
%\begin{enumerate}
%\item Let $F(\beta):=\{\alpha^{-X}\mid \beta\models \alpha^{-X}\}$. Please note that $F(\beta)$ must be DHORN
%\item If $F(\beta)\not\models\beta$ then return false.
%\item Suppose $F(\beta)\models\beta$.
%\item $K_0:=\{\alpha\mid \alpha^{-X}\mbox{ contains a negative clause}\}$
%\item $K_1:=\{\alpha\mid \alpha^{-X}\in F(\beta)\}$
%\item If $y$ occurs in $\beta$ positively, and occurs in a negative clause of some $\alpha^{-X}$, then $\neg y\rightarrow \alpha^X$.
%\item Let $E$ be the set of all formulas like $\neg y\rightarrow \alpha^X$.
%\item The existence of desired $K$ iff $(K_0\cup K_1)\models E$, which is a coNP problem
%\end{enumerate}
%\color{black}

\ \\

Ad 2: 2-Term $\Rightarrow_{conf}$ Term:\\
The computational complexities follow directly from the constructions for the $\models$-variants (remarks in the proof
of Theorem 1, case 3).
\end{proof}

---------------------------------------------------------------------
\begin{proof}
AD 3.1:  HORN  $\Rightarrow_{conf}$ HORN for $\approx$: Let $C=\{\alpha_1, \ldots, \alpha_n\}$ be the set components and $\beta$ the target formula. The test whether the target formula is satisfiable costs at most linear time. If that is not the case then
the configuration problem has no solution. Otherwise, we compute
$F(C, \beta) :=\{\alpha_i : \beta \models \alpha_i, 1 \leq i \leq n\}$. If $F(C, \beta)$ is empty then there is no solution. Otherwise, the set is satisfiable an we have
$\exists K \subseteq C: K \in \mbox{SAT and } K \approx \beta$ iff $F(C, \beta) \approx \beta$.\\
The direction from right to left is obvious (set $K= F(C, \beta)$). From left to right let $K \subseteq F(C,\beta)$
be a solution with the maximal number of components. If $F(c,\beta)$ is not a solution then there is some
$\alpha_i \in F(C, \beta)$, but not in $K$. Since $\beta \models \alpha_i$ and $K \approx \alpha_i$ we obtain
$K \models \alpha_i$ and therefore $K \cup\{\alpha_i\} \approx \beta$ in contradiction to the maximal
number of components.

3.2 HORN  $\Rightarrow_{conf}$ HORN for \AEQ and for \AEQB:\\
Let $C=\{\alpha_1, \ldots, \alpha_n\}$ a set of Horn components with helping variables $X$ and $\beta$ be the target formula with helping variables $Y$ over the free variables $V$.
The problem $\exists K \subseteq C: K \in  \mbox{ SAT and } (\exists X \alpha \approx (\exists Y \beta)$ is in $\Sigma^P_2$, because
the problem $(\exists X \alpha \approx (\exists Y \beta)$ is equivalent to the restricted equivalence for Horn formulas,
which is coNP-complete \cite{KBL}. The first part to select a satisfiable subset $K$ is in NP. \\

For \AEQ the completeness is open.
Next, we show the $\Sigma^P_2$-hardness for \AEQB y a reduction to the satisfiability problem for
closed $\exists \forall DNF$-formulas.
Let $\Phi:=\exists y_1,\cdots, y_k \forall x_1,\cdots, x_m \phi$ be a formula with DNF-kernel
$\phi = c_1 \vee \ldots \vee c_n$, where $c_i$ is a conjunction of literals.\\
For each variable $z$, we introduce a new variable $g(\neg z)$ and define $g(z)=z$.

Furthermore, we define $\psi_1:=\left(\bigvee_{i=1}^m (\neg x_i\vee \neg g(\neg x_i))\right)$.\\

Then, $\Phi$ is true iff $\exists \pi(y_1),\cdots, \pi(y_k) \forall x_1,\cdots, x_m (\pi(\phi) \wedge \theta)$ is true.


Let $$\psi_0:=\left(\bigwedge_{j=1}^m (\neg \pi(\neg x_j)\vee \neg x_j)\right) \mbox{ and }\theta:=\left(\bigvee_{i=1}^m (\neg x_i\wedge \neg\pi(\neg x_i))\right)$$

Please note that $\theta$ is not HORN  when $n>1$. For new variables $a_0, \ldots, a_k$ there exists an equivalent Horn formula
$$\theta \approx \exists a_0 \ldots \exists a_k (\neg a_k \wedge a_0 \wedge \bigwedge_{1 \leq i \leq k-1} ((\neg a_i \vee \neg x_i \vee a_{i+1}) \wedge (\neg a_i \vee \neg g(\neg x_i) \vee a_{i+1}))).$$

For a new variable $U$ we associate to $\Phi$ the set of components $C$:\\

Let $C_0:=\{\psi_0\wedge\psi_1 \wedge\left( \bigwedge_{i=1}^n(\pi(c_i)\rightarrow \neg U\vee \theta)\right)\}$ be consist of one component and \\

$$C:= C_0  \cup \{y_i: 1 \leq i \leq k\} \cup \{g(\neg y_i): 1 \leq i \leq k\}$$
with helping variables $Y := \{y_1, \ldots y_k, a_0, \ldots, a_k, g(\neg y_1), \ldots, g(\neg y_k)\}$\\
and the target formula \\
$\beta := (\neg U\vee\theta)\wedge \psi_0$ with helping variables $A := \{a_0, \ldots, a_k\}$ \\
The set of free variables is $V:=\{U\}\cup\{x_1,\cdots, x_m,g(\neg x_1),\cdots, g(\neg x_m)\}$

Now we will show

$$(\exists K\subseteq C: (\exists Y K) \approx \exists A \beta) \mbox{ if and only if} \Phi \mbox{ is true.}$$


Suppose $\Phi$ is true. Then there is a truth assignment
$v$ on $\{y_1,\cdots, y_k\}$, such that $\forall x_1,\cdots,x_m\varphi[\vec{y}/v(\vec{y})]$ is true, where $\vec{y}=y_1\cdots,y_k$. We define

$$K := C_0\cup \{y_i: v(y_i)=1, 1\leq i\leq k\}\cup \{g(\neg y_i): v(y_i)=0, 1\leq i\leq k\}$$

For any satisfying truth assignment $t$ for $\exists Y K$, if it makes $\theta$ true then $\beta$ is true under $t$. Suppose $t$ makes $\theta$ false, then $t$ corresponds a truth assignment on $x_1,\cdots, x_n$ accoding the truth of $x_i$ and $\pi(\neg x_i)$. Since $\Phi$ is true, some  $\pi(c_i)$ must be true under $t$, then $T(U)=1$. Therefore, $t(\beta)=1$.

Suppose $s$ is a satisfying truth assignment for $\exists A \beta$. Then $s*v$ satisfies $\psi_0\wedge \psi_1$. If $s$ makes $\theta$ true, then $s*v$ satisfies $\exists Y K$. Suppose $s(\theta)=0$. Then $s(U)=1$, hence $K$ is still satisfied by $s*v$.

Altogether, we have $(\exists Y K) \approx (\exists A \beta$).\\


For the inverse direction, suppose there is satisfiable $K\subseteq C$ such that $\exists Y K \approx \exists A \beta$.


Clearly, the formula in $C_0$ must be in $K$. Then by formula $\psi_1$, either $y_i$ or  $g(\neg y_i)$ is not in $K$.

Pick a truth assignment $e$ on $\{y_1,\cdots, y_k\}$ such that if $y_i\in K$ then $e(y_i)=1$, and $e(y_i)=0$  if else.

We need to show $\forall x_1,\cdots,x_m(c_1\vee\cdots\vee c_n)$ is true.

Consider any truth assignment $s$ on $\{x_1,\cdots, x_m\}$. Let $s'$ be the assignment defined by $s'(g(\neg x_i)=1$ iff $s(x_i)=0$. Then $e'*s'$ satisfies $\psi_0\wedge\psi_1$, where $e'$ is obtained from $e$ is the same way as $s'$.
We claim that, $(e'*s')$ makes $g(c_i)$ true for some $i$. Otherwise, we could set $U$ to be false,  and get a truth assignment satisfying $K$ but bot satisfying $\beta$ (please note that $s'(\theta)=0$), contradict the equivalence.

Consequently, $\Phi$ is true.
 \end{proof}


--------------------------------------------------------------------

\begin{proof} Ad 4: CLAUSE $\Longrightarrow_{conf}$ LITERAL\\
4.1: coNP-c for $\approx$):\\
Let $C=\{(\alpha_1 \vee L, \ldots, (\alpha_n \vee L), (\sigma_1 \vee \neg L), \ldots, (\sigma_r \vee \neg L), \pi_1, \ldots, \pi_t\}$ be a set of components, where $L$ does not occur in $\pi_j$. Furthermore, let $\beta=L$ be the target Literal. Then we have:\\
$\exists K \subseteq C: K \in$ SAT and $K \approx L$ iff $\{\alpha_1 \vee L), \ldots, (\alpha_n \vee L)\} \approx L$
iff $\alpha_n \wedge \ldots \wedge \alpha_n \in \overline{\mbox{SAT}}$. This shows that the problem is in coNP and because of the coNP-completeness of $\overline{\mbox{SAT}}$-problem that the configuration problem is coNP-hard, too.



4.2: $\Sigma^P_2$-hard for (\AEQ) and (\AEQB):\\
If the target formula is a literal, then the configuration problem can be reduced to the problem for \AEQ.
We will show the $\Sigma^p_2$-hardness by an reduction to the satisfiability problem for
$\Phi= \exists X \forall Y \phi$ where $\phi = \bigvee_{1 \leq i \leq r} \phi_i$ is a DNF-formula.\\
For a new variable $z$ we associate to $\Phi$ a set of components
$C =\{(\neg \phi_i \vee z): 1 \leq i \leq r\} \cup \{x_1, \neg x_1, \ldots, x_n, \neg x_n\}$, the helping variables
$X= \{x_1, \ldots, x_n\}$ and the target formula $\beta=z$.
Now we will show
$$\Phi \mbox{ is true iff }  \exists K \subseteq C: (K \in \mbox{  SAT and } (\exists X K) \approx z)$$


From right to left: Suppose, $K$ is satisfiable with $\exists X K \approx z$. Then $K$ contains some clauses with $z$ and eventually some $x_i$ and $\neg x_j$. Since $K$ is satisfiable no complementary pair $x_i$ and $\neg x_i$ is in $K$.
Let $K= \{(\neg \phi_{i_1} \vee z), \ldots, (\neg \phi_{i_s} \vee z), x_1^{\epsilon_1}, \ldots, x_t^{\epsilon_t}\}.$
Then we have
$\exists X K \approx z$ \\
$\Longrightarrow$
$z \approx \exists x_{t+1} \ldots \exists x_n \bigwedge_{1 \leq j \leq s} (\neg \phi_{i_j} \vee z)[x_1/\epsilon_1, \ldots, x_t/\epsilon_t]$\\
$\Longrightarrow$
$\exists x_{t+1} \ldots \exists x_n \bigwedge_{1 \leq j \leq s} (\neg \phi_{i_j} \vee z)[x_1/\epsilon_1, \ldots, x_t/\epsilon_t]\in
\overline{\mbox{SAT}}$\\
$\Longrightarrow$
$\forall x_{t+1} \ldots \forall x_n \bigvee_{1 \leq j \leq s} \phi_{i_j}[x_1/\epsilon_1, \ldots, x_t/\epsilon_t]
\in
\mbox{TAUT}$\\
$\Longrightarrow$
$\forall x_{t+1} \ldots \forall x_n \forall Y \bigvee_{1 \leq j \leq s} \phi_{i_j}[x_1/\epsilon_1, \ldots, x_t/\epsilon_t]$
is true\\
$\Longrightarrow$
$\exists x_1 \ldots \exists x_t \forall x_{t+1} \ldots \forall x_n \forall Y \bigvee_{1 \leq j \leq s} \phi_{i_j}$
is true\\
$\Longrightarrow$
$\exists X \forall Y \bigvee_{1 \leq j \leq r} \phi_{j}$
is true\\
$\Longrightarrow$ $\exists X \forall Y \phi$ is true.\\

From left to right:
Suppose, $\Phi$ is true for the truth assignment $x_1= \epsilon_1, \ldots, x_n= \epsilon_n$ for the existential variables. For the satisfiable subset of components
$K=\{ (\neg \phi_1 \vee z), \ldots, (\neg \phi_r \vee z), x_1^{\epsilon_1}, \ldots, x_n^{\epsilon_n}\}$.\\
we obtain:\\
$\exists X K \approx \bigwedge_{1 \leq j \leq r} (\neg \phi_j \vee z)[x_1/\epsilon_1, \ldots, x_n/\epsilon_n]
\approx (\bigwedge_{1 \leq j \leq r} \neg \phi_j)[x_1/\epsilon_1, \ldots, x_n/\epsilon_n] \vee z
\approx  \neg (\bigvee_{1 \leq j \leq r} \phi_j[x_1/\epsilon_1, \ldots, x_n/\epsilon_n]) \vee z$\\
$\approx \neg (\exists X \forall Y \phi) \vee z$ $\approx z$\\



\end{proof}

-----------------------------------------------------------------------------------------



\begin{proof}
Ad 5: CNF $\Rightarrow_{conf}$ CNF:
5.1  $\approx$ ($P^{NP[\text{log}\, n]}$, $D^P$-h)\\
For this problem we have no completeness result. The upper bound is $P^{NP[\text{log}\, n]}$.
For example, we can encode the minimal unsatisfiability problem for CNF formulas as follows:\\
Let $\alpha= \alpha_1 \wedge \ldots \wedge \alpha_n$ be a formula in CNF.  For $1 \leq i \leq n$ we introduce a renaming, which substitutes every variable $y$ in $\alpha$ with $y^i$. The renaming of the clause $\alpha_j$ is denoted as $\alpha^i_j$.
For a new variable $x$ we associate to $\alpha$ the set of components
$C_0 =\{(\alpha_1 \vee x) \wedge \ldots \wedge (\alpha_n \vee x)\}$ and for $1 \leq i \leq n$ we introduce the set of components
$ C_i=\{\alpha^i_j : 1 \leq j \leq n, j\not = i\}$. Then for the target formula $\beta = x \wedge C_1 \wedge \ldots \wedge C_n$ it holds\\
$\alpha$ is minimal unsatisfiable if and only if $\exists K \subset C_0 \cup \bigcup_{1 \leq i \leq n} C_i:
K \in$ SAT and $K \approx \beta$.

If $\alpha$ is minimal unsatisfiable then $\alpha $ is unsatisfiable and therefore $C_0 \approx x$. Since
after the deletion of a clause in $\alpha$ the resulting formula is satisfiable, $K = C_0 \cup \bigcup_{1 \leq i \leq n} C_i$ is satisfiable and equivalent to $\beta$. For the other direction let $K$ be a solution. Since $K \models x$
and $K$ is satisfiable, we see that $\alpha$ is  unsatisfiable. Moreover, because $C_i$ is part of the target
formula after the deletion of a clause in $\alpha$ the formula is satisfiable for every $1 \leq i \leq n$. Hence,
$\alpha$ is minimal unsatisfiable.\\

5.2 \AEQ and \AEQB ($\Sigma^p_3$-complete) for term\\

Since the problem of determining whether $F\equiv_V G$  is in $\Pi_2^P$ (for the complementary problem: we guess a clause $\gamma$ over $V$, check that $G\models \gamma$ but $F\not\models \gamma$, or $G\not\models\gamma$ but $F\models\gamma$).

Thus, the  configuration problems are in $\Sigma_3^P$, because additionally we have to guess a satisfiable $K\subseteq C$.\\
Next we shall show the hardness.

Consider $\Phi:=\exists \vec{x}\forall\vec{y}\exists \vec{z} \varphi$ where $\varphi$ is a CNF formula. \\

We assume w.l.o.g. that $\varphi$ contains a non-tautological clause over $\vec{z}$ (otherwise, we pick new varaiable $z'$ and consider $\exists\vec{x}\forall\vec{y}\exists\vec{z}\exists z'(\varphi\wedge z'$)).

By this assumption, $\forall\vec{x}\exists\vec{y}\exists\vec{z}\neg \varphi$ is true\\

We assume that each clause contains a positive occurrence of a variable from $\vec{y}\cup\vec{z}$. If otherwise pick a new variable $u$, and consider $\exists \vec{x}\forall\vec{y}\forall u\exists \vec{z}\varphi^u$, where $\varphi^u$ is obtained from $\varphi$ by adding $u$ to clauses containing no positive literal from $\vec{y}\cup\vec{z}$. Clearly, $\Phi$ and the resulting formula has the same truth.

By this assumption, $\forall\vec{x}\exists\vec{y}\exists\vec{z}\phi$ is true. (we will use this later).\\

$\varphi$ can be written as $(c'_1\vee c''_1)\wedge\cdots\wedge (c'_n\vee c''_n)$ in which $c'_i$ is over $\vec{x}$, while $c''_i$ is over $\vec{y}\cup\vec{z}$.

Pick new variables $w_1, \cdots, w_n,w$. Let

Let $$\psi:= \left(\bigwedge_{i=1}^n (c'_i\rightarrow w_i)\right)\wedge
\left(\bigwedge_{i=1}^n(\neg c'_i\wedge \neg c''_i\rightarrow \neg w_i)\right)$$

Let $$C_0:=\{\psi\wedge \neg c'_1\wedge  (c''_1\rightarrow w_1) , \cdots, \psi\wedge\neg c'_n\wedge (c''_n\rightarrow w_n)\}$$

$$C_1=\{c'_1, \cdots, c'_n, \}$$

$$C_2=\{x_1,\neg x_1\cdots, x_n,\neg x_n\}$$

$$C_3=\{(w\rightarrow (c'_1\vee c''_1)\wedge\cdots\wedge (c'_n\vee c''_n))\}$$

Now let $$C:=C_0\cup C_1\cup C_2\cup C_3\cup\{w\}$$

Let $\beta$ be $$ w_1\wedge\cdots\wedge w_n\wedge w$$

Let $V:=\vec{y}\cup\{w_1,\cdots,w_n,\}\cup\{w\}$\\


%Claim: Suppose $K\subseteq C$ satisfiable, $c_i\not \in K$ and $K\not\models c_i$, %then
%$K\cup\{\neg c_i, \psi\wedge (c''_i\rightarrow w_i)\}$ is still satisfiable.


(Lemma: $F\models_V G$ iff for for any truth assignment $t$ on $V$, if it can be extended to a satisfying truth assignment for $F$, it can be extended to an satisfying truth assignment for $G$.
) \\

We shall show $\Phi$ is true if and only if there is satisfiable $K\subseteq C$ s.t. $K\equiv_V \beta$\\



\color{red} From right to left:\color{black}

Given a such $K\equiv_V \beta$.
Clearly, $K\cap C_0$ is non-empty.\\



It must be that $w\in K$.
Let $t$ satisfy $K$.
By our assumption we know $t\uparrow\vec{x}$ can be extended to $s$ which falsifies $\varphi$. Now we assign truth values to $w_i$ according to the tuth values $s(c'_i)$ and $s(c''_i)$ to make formulas $\psi$ and  $(c''_i\rightarrow w_i)$ to be true. There  must be some $w_i$ which is false. Then if we set $w$ to be false, $K$ is still be satisfied by $s$ (because clauses in $C_3$ are satisfied). This contradict the $V$-euivalence.\\

It must be that $C_3\subseteq K$, i.e., $w\rightarrow (c'_1\vee c_1'')\wedge\cdots\wedge (c'_n\vee c''_n)$ is in $K$. Suppose it is not the case. Similar as above, there would be a satisfying truth assigment of $K$ which makes some $w_i$ false, contradicts the $V$-equivalence. \\


Suppose $K\not\models c'_i$. Then $\psi\wedge\neg c'_i \wedge (c''_i\rightarrow w_i)$ must be in $K$.
Otherwise, $K\cup\{\neg c'_i\}$ is consitent,  then there is a satisfying truth assignment $t$ for $K$ such that $t(c'_i)=0, t(w)=1, t(w_i)=0$. (In fact by our assumption, $t$ can be assumed to satisfiy $(c'_1\vee c''_1)\wedge\cdots\wedge (c'_n\vee c''_n)$ because every clause contains positive literal from $\vec{y}\cup\vec{z}$). This contradict the fact that $K\equiv_V\beta$.


Now we can see for each $i$, either $K\models \neg c'_i$ or $K\models c'_i$.
That is, for any two satisfying truth assignments $t_1, t_2$ of $K$, each $c'_i$ has the same truth under $t_1\uparrow \vec{x}$ and $t_2\uparrow\vec{x}$.  \\





Now fix a trut assigment $e$ on $\vec{x}$ which can be extended to a satisfying truth assigment of $K$.

Consider any truth assigment $s$ on $\vec{y}$. Since $s$ can be extended to satisfy $\beta$, it can be extended to a truth assigment $t$ which satisfies $K$. Please note we can assmue that $t\uparrow\vec{x}$ is $e$.
Since $w$ and $w\rightarrow (c'_1\vee c''_1)\wedge\cdots\wedge (c'_n\vee c''_n)$ are in $K$. We can see $t$ satisfy $\varphi$.

Consequently, for any truth assigment $s$ on $\vec{y}$, $e*s$ can be extended to a satisfying truth assigment of $\varphi$. Thus, $\Phi$ is true.


\ \\

{\color{red} From left to right}.\color{black}


Suppose $\Phi=\exists\vec{x}\forall\vec{y}\exists\vec{z}\varphi$ is true.

Let $e$ be a truth assignment on $\vec{x}$ such that $\forall\vec{y}\exists\vec{x}\varphi[x/e]$ is true


Let $$K_0:=\{\psi\wedge \neg c'_i\wedge (c''_i\rightarrow w_i)\mid e(c'_i)=0, i=1,\cdots,n\}$$

$$K_1:=\{c'_i\mid e(c'_i)=1, i=1,\cdots,n\}$$

$$K_2:=\{x_i\mid e(x_i)=1, i=1,\cdots, n\}\cup\{\neg x_i\mid e(x_i)=0, i=1,\cdots,n\}$$

$$K:=K_0\cup K_1\cup K_2\cup C_3\cup\{w\}$$

Consider any truth assignment $s$ on $V$.

Suppose $s$ can be extended to satisfy $K$. Say the extenson is $t$. Since $w\in K$, $(c'_1\vee c''_1)\wedge\cdots\wedge(c'_n\vee c'_n)$ be be true under $t$. By formulas in $K_0$, we can see each $w_i$ is true under $t$. That means $s$ make $\beta$ true. Thus, $K\models_V\beta$. \\

Now suppose $s$ satisfies $\beta$. Since $\Phi$ is true, $e*(s\uparrow\vec{y})$ can  be extended to satisfy $\Phi$. Let $t$ be such an extension. Next we show $t$ can be extended to satisies $K$. Since $t$ makes $\varphi$ true, either $c'_i$ is true or $c''_i$ is true. We set each $w_i$ to be true. Then all clauses in $K_0$ are true. Set $w$ to be true. Then clauses in $C_3$ are true. Please note formulas in $K_1\cup K_2$ are already atisfied by $e$. Consequently, $s$ can be extended to satisfy $K$.
Hence, $\beta\models_V K$.\\

Altogether we obtain $K\equiv_V \beta$.
\end{proof}
\end{document}


